\documentclass[12pt,a4paper]{article}
\usepackage[russian]{babel} %russian
\rmfamily

\usepackage{hyperref} % references
\usepackage{amsfonts} % beautiful F letter

\usepackage{xcolor}
\usepackage{amsmath}
\usepackage{amsthm}
\usepackage{makecell}
\usepackage{amssymb}
\usepackage{listings}
\usepackage{comment}
\usepackage{tikz}

\textwidth=16.5cm
\textheight=24.5cm
\parindent=10.0mm

\hypersetup{
	colorlinks=true,
	linkcolor=blue,
	citecolor=blue,
	urlcolor=blue,
	filecolor=blue,
	pdfpagemode=None,
	pdfstartview=FitH,
	pdfhighlight=/N
}

\usepackage[left=2cm,right=1.5cm,top=2cm,bottom=2cm]{geometry}

\linespread{1.3}

\begin{document}
	
	\newpage
	\tableofcontents
	\setcounter{page}{2}
	\newpage
	
	\section{Предмет и специфика философского знания. Место философии в системе культуры~\checkmark}
	\par Термин <<философия>> в переводе с древнегреческого означает любовь к мудрости, а с древ­неиндийского (<<даршака>>) — видение истины. Слово <<философ>> впервые употребил греческий математик и мыс­литель Пифагор по отношению к людям, стре­мящимся к интеллектуальному знанию и правильному образу жизни. 
	\par Истолкование и закрепление в европейской культуре термина <<фи­лософия>> связано с именем Платона. Возникновение философии в VI в. до н. э. означало постепенный переход людей от мифа к самостоятельному размышлению о мире, о челове­ческой судьбе, стремление найти истину, стремление к мудрости. Поэтому философия выступала как <<наука о всеобщем, которая исследует сущее само по себе, первые начала и причины>> (Аристотель). 
	\par  Философия — форма духовной деятельности, направленная на по­становку, анализ и решение коренных мировоззренческих вопросов, связанных с выработкой целостного взгляда на мир и на место чело­века в нем. В настоящее время философия представляет собой и науку о все­общих законах развития природы, общества, мышления, познания и особую форму общественного сознания, теоретическую основу ми­ровоззрения, систему философских дисциплин, способствующих фор­мированию духовного мира человека. 
	\par Предметом философии называется круг вопросов, которые изуча­ет философия. Общую структуру предмета философии, философско­го знания составляют четыре основных раздела: онтология -- учение о бытии; гносеология (эпистемология) -- учение о познании; человек; общество.
	\par По сравнению с другими формами знания, философия обладает следующими специфичными характеристиками:
	\begin{enumerate}
		\item Фундаментальность — философия занимается первоосновами бытия и знания.
		\item Критичность — она подвергает сомнению даже очевидные основания (например, Декарт: "Cogito, ergo sum").
		\item Всеобщность — её суждения претендуют на универсальность.
		\item Полисемичность — философские понятия многозначны и подвержены интерпретации.
	\end{enumerate}
	Философия занимает особое место в системе культур человека, выступая
	\begin{enumerate}
		\item как источник норм и ценностей (этика, политическая философия);
		\item как методологическое основание для других наук (эпистемология, философия науки);
		\item как историко-критическая форма сознания, осмысляющая культуру в целом.
	\end{enumerate}
	Философия повлияла на формирование естественных и гуманитарных наук (например, из философии выделились физика, социология, психология), на право (естественно-правовая теория), политику (от Платона до Хабермаса), искусство (эстетика) и даже религию (через философскую теологию).
	
	
	
	\section{Общая характеристика и основные понятия древнеиндийской философии~\checkmark}
	
	\subsection{Периодизация}
	История древнеиндийской философии традиционно делится на два крупных периода:
	\begin{itemize}
		\item Ведический период: Характеризуется созданием Вед --- древних священных писаний индуизма, считающиеся божественным откровением. В широком смысле, "Веды" (санскр. знание) — это собрание текстов, а в узком смысле — сами тексты, называемые самхитами. Веды считаются источником истины и всеобъемлющего знания о мире, Боге и человеке. Существуют также более поздние комментарии к ним — Упанишады.
		В этот период закладываются основы будущих философских систем, формируются представления о Брахмане и Атмане.
		\item Классический (эпический) период: Этот этап связан с возникновением и развитием систематизированных философских школ, известных как даршаны.
	\end{itemize}
	
	
	\subsection{Основные школы (даршаны)} Даршаны делятся на две большие группы в зависимости от их отношения к авторитету Вед.
	\textbf{Ортодоксальные (астика)}
	
	Школы, признающие авторитет Вед:
	\begin{itemize}
		\item Санкхья: Радикально дуалистическая система, признающая два вечных и независимых начала: \textbf{Пуруша} (пассивное, чистое сознание, "Я") и \textbf{Пракрити} (активная, бессознательная, материальная первопричина всего сущего). Освобождение (мокша) достигается через осознание отличия Пуруши от Пракрити.
		\item Йога: Во многом принимает метафизику Санкхьи, но добавляет концепцию \textbf{Ишвары} (Бога-творца). Основное внимание уделяется практическим методам достижения освобождения через контроль над умом и телом. Центральным текстом является "Йога-сутры" Патанджали, описывающий восьмеричный путь к самадхи (состоянию просветления).
		\item Ньяя: Школа логики и эпистемологии. Основной вклад --- разработка учения о четырех источниках достоверного знания (\textit{праманах}): прямое восприятие (\textit{пратьякша}), логический вывод (\textit{анумана}), сравнение (\textit{упамана}) и авторитетное свидетельство/слово (\textit{шабда}).
		\item Вайшешика: Атомистическое учение, согласно которому весь физический мир сводится к комбинациям вечных и неделимых атомов (\textit{параману}). Разработала учение о категориях бытия (\textit{падартхах}), включающих субстанцию, качество, действие, общее, особенное и присущность.
		\item Миманса: Основная цель --- экзегетика (толкование) Вед и обоснование необходимости ритуалов. Разработала концепцию \textbf{апурвы} --- невидимой силы, которая возникает в результате правильно выполненного ритуала и приносит плоды в будущем.
		\item Веданта: Одна из наиболее известных школ, в частности ее направление Адвайта-веданта, основанное Шанкарой. Центральная идея адвайты --- "недвойственность", тождество индивидуальной души (Атмана) и высшей реальности (Брахмана).
	\end{itemize}
	
	\textbf{Неортодоксальные (настика)}
	
	Школы, не признающие авторитет Вед:
	\begin{itemize}
		\item Буддизм: Основан на учении Сиддхартхи Гаутамы (Будды). Ключевые идеи: учение о "четырех благородных истинах", концепция "взаимозависимого происхождения" и отрицание существования вечной души (анатмавада). Конечная цель --- достижение нирваны.
		\item Джайнизм: Радикальная этическая система. Главные принципы: \textbf{Ахимса} (ненасилие по отношению ко всем живым существам), \textbf{Апариграха} (непривязанность, нестяжание) и \textbf{Анекантавада} (учение о многогранности истины, отрицание однобокого взгляда на реальность).
		\item Чарвака-локаята: Материалистическая и скептическая школа. Признает только прямое восприятие как источник знания, отрицает существование души, Бога, кармы и перерождения. Считает, что мир состоит из четырех элементов (земля, вода, огонь, воздух), а сознание является продуктом их соединения.
	\end{itemize}
	
	\subsection{Ключевые понятия}.
	Центральные концепции индийской философии определяют ее этическую и сотериологическую (учение о спасении) направленность.
	\begin{itemize}
		\item \textit{Брахман} и \textit{Атман}: В Упанишадах и Веданте Брахман --- это высшая, безличная, единая реальность.  Атман --- индивидуальная душа, "Я".
		\item \textit{Сансара}: Круговорот рождений и смертей, в который вовлечены все живые существа. Это мир страданий и непостоянства.
		\item \textit{Карма}: Закон, согласно которому поступки (как физические, так и ментальные), совершенные в этой жизни, определяют условия последующих перерождений.
		\item \textit{Мокша} (или \textit{Нирвана} в буддизме): Высшая цель духовной практики, означающая освобождение из круговорота сансары и связанных с ней страданий.
	\end{itemize}
	
	\section{Конфуцианство и даосизм как течения древнекитайской философии~\checkmark}
	
	\subsection{Конфуцианство} Конфуцианство--- это этико-философское учение, которое возникло в Китае на рубеже VI–V веков до н. э. и оказало огромное влияние на развитие китайской цивилизации. Основоположник --- Конфуций.
	
	\textbf{Ключевые идеи}
	\begin{itemize}
		\item \textit{Жэнь} (гуманность): Человеколюбие, милосердие, уважение к другим. Основополагающая категория конфуцианской этики.
		\item \textit{Ли} (ритуал): Правила поведения, этикет, церемонии. Соблюдение \textit{ли} является внешним проявлением \textit{жэнь} и обеспечивает гармонию в обществе и государстве.
		\item \textit{И} (долг/справедливость): Моральное обязательство поступать правильно и по справедливости, исходя не из личной выгоды, а из чувства долга. \textit{И} уравновешивает \textit{жэнь}, придавая ему осмысленность и направленность.
		\item \textit{Цзюнь-цзы} (благородный муж): Идеальная личность в конфуцианстве, которая воплощает в себе гуманность (\textit{жэнь}) и следует ритуалу (\textit{ли}). \textit{Цзюнь-цзы} стремится к моральному самоусовершенствованию и служит примером для других.
		\item \textit{Сяо} (сыновья почтительность): Уважение и преданность по отношению к родителям и старшим. Считается основой всех остальных добродетелей и социального порядка.
	\end{itemize}
	
	\subsection{Даосизм} Даосизм --- одно из главных направлений китайской философии, возникшее наряду с конфуцианством. Основоположник --- Лао-цзы.
	
	\textbf{Ключевые идеи}
	\begin{itemize}
		\item \textit{Дао} (путь): Центральное понятие, обозначающее невыразимый в словах, всеобщий закон и первооснову мира. Это естественный ход вещей, которому все подчинено.
		\item \textit{Дэ} (благодать, добродетель): Конкретное проявление \textit{Дао} в каждой отдельной вещи. Это внутренняя сила или жизненная энергия, которую вещь получает от \textit{Дао}.
		\item \textit{У-вэй} (недеяние): Один из важнейших принципов даосизма, который означает не бездействие, а деятельность, согласованную с естественным ходом вещей (\textit{Дао}). Это спонтанное, гармоничное действие без насилия над природой вещей.
	\end{itemize}
	

	\section{Досократовский период древнегреческой философии~\checkmark}	
	Досократовский период (VII–V вв. до н.э.) — это начальный этап древнегреческой философии, предшествующий Сократу. Основная особенность этого периода — переход от мифологического объяснения мира к рациональному, натурфилософскому мышлению.
	
	Философия ещё тесно связана с натурфилософией, то есть поиском первооснов (архэ -- первооснова, первовещество, первоэлемент, из которого состоит мир. Для характеристики учений первых философов этот термин использовал Аристотель) всего сущего. Основной вопрос: что лежит в основе мира?, каким образом из этого начала возникает многообразие вещей?
	
	Философия возникает в греческих полисах Малой Азии (Иония) и Южной Италии (Великая Греция) — регионах активной торговли, культурного обмена и политической свободы, что способствовало интеллектуальному развитию.
	
	\subsection{Милетская школа}
	Представители: Фалес, Анаксимандр, Анаксимен
	
	\begin{enumerate}
		\item Фалес (VII–VI вв. до н.э.): первым ввёл понятие архэ как природной субстанции — вода. Он рассматривал природу как живую и одушевлённую.
		\item Анаксимандр: предложил в качестве первоосновы апейрон (неопределённое, беспредельное), подчеркивая абстрактный и неосязаемый характер начала. Он первым попытался описать эволюцию живого из первичной среды.
		\item Анаксимен: вернулся к конкретной субстанции — воздуху как архэ, объясняя изменения через сгущение и разрежение.
	\end{enumerate}
	Фалес и Анаксимен опирались на конкретные природные стихии, в то время как Анаксимандр впервые предложил абстрактный принцип. Это шаг от мифологии к философской абстракции.
	
	\subsection{Пифагорейская школа}
	Представитель: Пифагор и его последователи
	
	\begin{enumerate}
		\item Архэ — число, космос — это гармония и порядок, подчинённый математическим законам.
		
		\item Пифагорейцы первыми соединили философию, математику, этику и религиозную практику.
		
		\item Космос воспринимался как живое одушевлённое целое, подчинённое логосу числовых соотношений.
	\end{enumerate}
	
	Пифагорейцы внесли в философию идею логической структуры бытия, что предвосхищает платоновскую онтологию и математику как основу наук.
	
	\subsection{Элейская школа}
	Представители: Парменид, Зенон
	\begin{enumerate}
		\item Парменид: бытие едино, неподвижно и вечно. Движение и множественность — иллюзия чувств, истину можно познать только разумом (логос).
		
		\item Зенон: предложил знаменитые апории против движения и множественности (Ахиллес и черепаха, стрела и т.д.), чтобы доказать позицию Парменида.
	\end{enumerate}
	Элейцы отвергали чувственные данные как достоверный источник знания, вводя различие между мнением (докса) и истиной (алетейя). Это делает их предтечами онтологии и эпистемологии.
	
	\subsection{Гераклит из Эфеса}
	\begin{enumerate}
		\item В противоположность Элейцам утверждал: всё течёт (панта рей), сущность бытия -- изменение и становление.
		
		\item Архэ -- огонь, как символ изменения.
		
		\item Принцип мира -- логос, универсальный закон, проявляющийся в борьбе противоположностей.
	\end{enumerate}
	Гераклит и Парменид занимают антиподные позиции: первый утверждает становление как бытие, второй — бытие как неподвижную вечность. Их спор стал основой для диалектики Платона и Гегеля.
	
	\subsection{Атомисты}
	Представители: Левкипп, Демокрит
	\begin{enumerate}
		\item Архэ — атомы и пустота.
		
		\item Всё существует как комбинация неделимых, неизменных частиц.
		
		\item Возникновение и разрушение — это всего лишь изменение конфигураций атомов.
	\end{enumerate}
	Атомизм предлагает раннюю форму механистического материализма и предвосхищает новоевропейскую физику (Ньютон, Декарт).
	
	В досократике формируется понятийный аппарат: архэ, космос, логос, апейрон, атом, бытие. Заложены основы онтологии, натурфилософии, логики, этики (у пифагорейцев). Присутствует тенденция к рационализации мира, отделению философии от мифа (впервые поиск истины не через богов, а через разум). Начинается переход от натурфилософии к философии человека, что будет реализовано Сократом.
	
	\par Сократ, Платон, Аристотель осмысливают и критикуют досократиков.
	Платон строит свою диалектику на полемике между Гераклитом и Парменидом.
	Аристотель в <<Метафизике>> систематизирует учения досократиков, рассматривая их как "поискователей архэ".
	Их идеи сохраняются и трансформируются в эллинистических школах (стоицизм, эпикуреизм) и позднейшей философии.
	
	
	\section{Философия Платона~\checkmark}
	Платон (427–347 гг. до н.э.) - основатель объективного идеализма, создатель академии — первой философской школы, где философия преподавалась как систематическая дисциплина. 
	\par Согласно учению Платона лишь мир идей представляет собой истинное бытие, а конкретные вещи - это нечто среднее между бытием и небытием, они только тени идей. Для объяснения этой идеи, Платон приводит аллегорию -- Миф о пещере. Люди с рождения прикованы в пещере, видят лишь тени объектов, проходящих за их спинами (чувственный мир). Один из узников освобождается, выходит из пещеры и впервые видит настоящий свет (мир идей), постигает истину и благо (солнце). Вернувшись, он пытается просветить других, но сталкивается с непониманием.
	\par Платон объявил мир идей божественным царством, в котором до рождения человека пребывает его бессмертная душа. Затем она попадает на грешную землю, где временно находясь в человеческом теле, как узник в темнице она вспоминает о мире идей. 
	\par Бытие - тождественное и неизменное, однако в диалогах <<Софист>> и <<Парменид>> Платон приходит к выводу, что высшие роды сущего - Бытие, движение, покой, тождество и изменение - могут мыслится только таким способом, что каждый из них и есть и не есть, и равен самому себе и не равен, и тождественен себе и приходит в иное. Поэтому бытие содержт в себе противоречия: оно едино и множественно, вечно и преходяще, неизменно и изменчиво. 
	\par Теория познания Платона опирается на его учение о душе. Платон считал, что человек, как телесное существо, смертен. Душа же его бессмертна. Когда человек умирает, его душа, по Платону, не погибает, а лишь освобождается от телесного покрова как своей темницы и начинает свободно путешествовать в поднебесной сфере. Во время этого путешествия она соприкасается с миром идей и созерцает их. Поэтому суть процесса познания состоит в припоминании душой того идей, которые она уже созерцала (анамнезис). Истинное знание дает только мышление. Мышление же это не зависимый от чувственных восприятий абсолютно самостоятельный процесс припоминания. Только мышление дает знание идей. Чувственное восприятие пораждает лишь мнения о вещах. Человеческая душа независима от тела и бессмертна. Душа состоит из 3 частей: разумной, кот создается самим творцом (Демиургом), аффективной и вожделеющей, которые создаются низшими богами. 
	
	\section{Философия Аристотеля~\checkmark}
	Среди учеников Платона выделился гениально одаренный мыслитель Аристотель (384-322 гг до н.э.), создавший самобытное философское учение - одно из величайших в древнегреческой философии. Родился в Фракийском городе Стагир, обучался в платоновской академии. Обширное литературное наследство Аристотеля дошло до нас не полностью, и не все в нем составляют тексты самого Аристотеля. Весьма важны для понимания учения Аристотеля такие его сочинения, как <<О душе>>, <<Физике>>, <<Категории>>, <<Никомахова этика>>. 
	\par Философское учение Аристотеля сложилось в тесной связи с естественно-научными и социально-политическими исследованиями, которые он проводил вместе со своими учениками. В круг интересов Аристотеля входят вопросы логики, психологии, теории познания, учения о бытии, космологии, физики, зоологии, политической экономии, политики, этики, педагогики, риторики, эстетики. 
	\par Учение Аристотеля - объективный идеализм, правда непоследовательный, включающий в себя ряд, по существу, материалистических положений. Это учение сложилось в результате критики учения Платона об идеях. По Аристотелю, каждая единичная вещь есть единство материи (гиле -- исходный материал, первоматерия) и формы (морфе -- форма или образ, определяющий сущность и структуру вещей). Форма нематериальна, но она не есть некая потусторонняя сущность, извне привходящая в материю. Так, медный шар есть единство вещества - меди - и формы - шаровидности, которая придана меди мастером, но в реально существующем шаре она составляет одно с веществом. 
	\par Противоположность формы и материи, по Аристотелю, небезусловная. Медь есть материал по отношению к шару, который из нее отливается, но та же медь - <<форма>> по отношению к физическим элементам, соединение которых составляет медь. Медь лишена <<формы>>, поскольку меди еще не придана форма шара. И та же медь есть возможность формы, поскольку медник может внести в вещество меди эту форму. <<Форма>> есть действительность того, возможностью чего является <<материал>>, и, наоборот, <<материал>> есть возможность того, действительностью чего будет форма. 
	\par Согласно Аристотелю, в пределах мира чувственно воспринимаемых вещей возможен последовательный переход от <<материи>> к соотносительной ей <<форме>>, от <<формы>> - к соотносительной ей <<материи>>. Поднимаясь по лестнице <<форм>>, мы, утверждает Аристотель, доходим наконец до высшей <<формы>>, которую уже нельзя рассматривать как <<материю>> или как <<возможность>> более высокой <<формы>>. Такая предельная <<форма>> есть перводвигатель, или Бог, пребывающий вне мира. 
	\par В то время как у Платона вещи чувственно воспринимаемого мира рассматриваются лишь как видимость, искаженное отражение истинно сущего, у Аристотеля чувственно воспринимаемая вещь рассматривается как реально существующее единство <<формы>> и <<материи>>. Таким образом, по Аристотелю, сущность вещи (ее <<форма>>) неотделима от того, сущностью чего она оказывается. Если Платон утверждал, что чувственно воспринимаемые вещи не могут быть предметом познания, то для Аристотеля окружающий человека мир - это и есть то, что познается, изучается и благодаря чему достигается познание общего. Столкновение материалистических и идеалистических подходов проявляется и в Аристотелевой концепции причинности. Для объяснения того, что существует он различает четыре основных вида причин:
	\begin{enumerate}
		\item сущность и суть бытия, в силу которой всякая вещь такова, какова она есть (формальная причина);
		\item материя и подлежащее (субстрат) - то, из чего что-либо возникает (материальная причина);
		\item движущая причина, начало движения;
		\item целевая причина -- то, ради чего что-либо осуществляется. 
	\end{enumerate}
	
	Хотя Аристотель признавал материю одной из первых причин и считал ее своего рода сущностью, однако в материи он видел только пассивное начало (только возможность стать чем-либо), всю же активность приписывал остальным трем причинам, причем сути бытия - форме - приписывал вечность и неизменность, а источником всякого движения считал неподвижное, но все движущее начало - бога. Движение, по Аристотелю, есть переход чего-либо из возможности в действительности. 
	\par Творчество Аристотеля является вершиной не только античной философии, но и всего древнего мышления. Он дал начало в том или ином смысле большинству последующих философских систем. Содержательность и разработанность философской системы Аристотеля были универсальны. Длительное время его определял как Философа с большой буквы.
	
	\section{Эллинистическо-римская философия~\checkmark}
	
	\subsection{Общая характеристика периода}
	Эллинистическая философия охватывает период с конца IV в. до н.э. до конца античного мира. Она возникает в условиях упадка греческих полисов и формирования огромных империй. Философия смещает свой фокус с проблем политического устройства на вопросы этики, индивидуального счастья и спасения души в изменчивом и часто враждебном мире.
	
	\subsection{Основные школы}
	\begin{itemize}
		\item \textbf{Стоицизм} (Зенон из Кития, Сенека, Эпиктет, Марк Аврелий):
		\begin{itemize}
			\item Основатель: Зенон из Кития основал школу в Афинах около 300 г. до н.э.
			\item Учение: Мир --- единый телесный космос, пронизанный божественным Логосом (разумом), который является судьбой.
			\item Этика: Цель жизни --- достижение \textbf{апатии} (бесстрастия) и жизнь в согласии с природой, то есть с Логосом. Сенека уделял особое внимание борьбе со страстями, считая их главной причиной несчастий. Эпиктет ввел "дихотомию контроля": мы должны разделять то, что находится в нашей власти (наши суждения, стремления, мнения), и то, что нет (наше тело, имущество, внешние события), и беспокоиться только о первом. Марк Аврелий в своих "Размышлениях" применял стоические принципы для сохранения внутреннего мира и исполнения долга в роли императора.
		\end{itemize}
		
		\item \textbf{Эпикуреизм} (Эпикур, Лукреций Кар):
		\begin{itemize}
			\item Учение: Физика Эпикура атомистична. Боги существуют, но безразличны к миру, душа смертна.
			\item Этика: Высшее благо --- удовольствие, понимаемое как отсутствие телесной боли (\textbf{апония}) и душевных тревог (\textbf{атараксия}). Эпикур призывал к умеренным и разумным удовольствиям, избегая крайностей.
		\end{itemize}
		
		\item \textbf{Скептицизм} (Пиррон, Секст Эмпирик):
		\begin{itemize}
			\item Учение: Пиррон считал, что мы не можем ничего знать о реальной природе вещей, поэтому следует воздерживаться от любых окончательных суждений (\textbf{эпохе}). Секст Эмпирик систематизировал аргументы скептиков (т.н. "тропы"), доказывающие невозможность достоверного знания.
			\item Этика: Цель --- достижение \textbf{атараксии} (невозмутимости) через воздержание от суждений.
		\end{itemize}
		
		\item \textbf{Неоплатонизм} (Плотин, Прокл):
		\begin{itemize}
			\item Учение: Система строится вокруг иерархии бытия: сверхсущее \textbf{Единое} (Благо) эманирует (изливается), порождая \textbf{Ум} с миром идей, а затем --- \textbf{Мировую Душу}.
			\item Этика: Цель --- возвращение души к своей первопричине, Единому, через философское созерцание и мистический экстаз.
		\end{itemize}
	\end{itemize}
	
	\section{Средневековая европейская философия периода патристики~\checkmark}
	
	Патристика (от лат. \textit{pater} -- отец) -- это философия и теология Отцов Церкви, христианских мыслителей II--VIII веков. Данный период характеризуется становлением христианской догматики и философии, а также активной полемикой с языческой мыслью и гностицизмом. Этот процесс становления догматики был неразрывно связан с деятельностью Вселенских соборов (особенно Никейского и Халкидонского), на которых формулировались и утверждались ключевые догматы, такие как Символ веры и учение о двух природах Христа. Гностицизм представлял особую угрозу, так как предлагал альтернативную христианству эзотерическую трактовку Писания, утверждая, что материальный мир создан злым демиургом, а спасение доступно лишь избранным, обладающим тайным знанием (гнозисом).
	\subsection{Основные черты патристики:}
	\begin{itemize}
		\item Теоцентризм: Бог находится в центре философского осмысления мира и человека.
		\item Креационизм: Учение о сотворении мира Богом из ничего (\textit{creatio ex nihilo}). Это принципиально отличало христианство от античной мысли, где космос считался вечным, а материя — совечной богу.
		\item Провиденциализм: Понимание истории как осуществления божественного замысла о спасении человека. История обретает смысл и направленность от сотворения мира к Страшному суду.
		\item Персонализм: Утверждение абсолютной ценности человеческой личности, созданной по образу и подобию Бога и обладающей свободой воли.
		\item Синтез с античной философией: Идеи платонизма и неоплатонизма активно использовались для рационального обоснования христианского вероучения. Философия рассматривалась как <<служанка богословия>>, инструмент для прояснения истин Откровения.
	\end{itemize}
	
	\subsection{Ключевые представители и идеи:}
	\textbf{Ранний период (II–III вв.): Апологетика}
	
	В ранний период \textit{апологеты} (Юстин Мученик, Тертуллиан, Климент Александрийский) защищали христианство от критики со стороны язычников и римского государства.
	\begin{itemize}
		\item Тертуллиан выдвинул знаменитый тезис <<Верую, ибо абсурдно>> (\textit{Credo quia absurdum}), подчеркивая приоритет веры над разумом и парадоксальность христианских догматов. Он представлял фидеистическую линию, считая, что Афины (символ разума) не имеют ничего общего с Иерусалимом (символом веры).
		\item Климент Александрийский и Ориген (Александрийская школа), наоборот, стремились к их гармонии, считая философию путем к познанию Бога. Они заложили основы для систематического синтеза греческой мудрости и христианского откровения.
	\end{itemize}
	
	\textbf{Классический период (IV–V вв.): Систематизация учения}
	
	Это время систематизации учения, когда были сформулированы фундаментальные догматы христианства.
	\begin{itemize}
		\item На Востоке \textit{Великие каппадокийцы} (Василий Великий, Григорий Богослов, Григорий Нисский) разработали учение о Троице. Они ввели различение между сущностью (\textit{ousia}) и ипостасью (\textit{hypostasis}), что позволило сформулировать догмат о едином Боге в трех Лицах. Это различие стало краеугольным камнем православной теологии.
		\item Ключевой фигурой западной патристики является Аврелий Августин (Блаженный Августин). Его философия стала фундаментом для всей последующей западноевропейской мысли.
		\begin{itemize}
			\item Учение о Боге и творении: Бог — это высшее бытие, сотворившее мир и время. Время, по Августину, не существует вне души; оно есть <<растяжение души>> (\textit{distentio animi}) в ее трех модусах: настоящем как созерцании, прошлом как памяти и будущем как ожидании.
			\item Учение о познании: Истинное знание возможно не через чувства, а через божественное озарение (иллюминацию), когда Бог просветляет человеческий ум, позволяя ему видеть вечные истины.
			\item Проблема зла: Зло не субстанционально, а представляет собой лишь недостаток, умаление добра или отпадение от него. Оно есть результат неправильного использования свободной воли.
			\item Философия истории: В труде <<О граде Божьем>> история предстает как борьба двух <<градов>> --- земного (основанного на любви к себе до презрения к Богу) и небесного (основанного на любви к Богу до презрения к себе).
			\item Свобода воли и благодать: Человек обладает свободой воли, но его воля ослаблена первородным грехом. Поэтому его спасение невозможно без божественной благодати, которая направляет и исцеляет волю. Этот вопрос лег в основу его полемики с Пелагием.
		\end{itemize}
	\end{itemize}
	
	\section{Средневековая философия стран Западной Европы периода схоластики~\checkmark}
	
	Схоластика (от лат. \textit{schola} -- школа) -- систематическая <<школьная>> философия, преподававшаяся в монастырских школах и университетах в IX--XV веках. Ее главной задачей было рациональное обоснование догматов христианской веры. Знаменитая формула этого периода: <<Философия -- служанка богословия>>. Основной метод схоластики — \textit{диалектический}, включающий в себя анализ понятий, постановку вопросов (\textit{quaestio}), приведение аргументов <<за>> (\textit{pro}) и <<против>> (\textit{contra}) и формулирование окончательного вывода (\textit{determinatio}).
	\subsection{Спор об универсалиях:}
	Центральная проблема схоластики, касающаяся онтологического статуса общих понятий (универсалий). Например, существует ли <<человек вообще>> или только конкретные люди?
	\begin{itemize}
		\item Крайний реализм: Общие понятия существуют реально, до и независимо от единичных вещей, в уме Бога (Ансельм Кентерберийский). Позиция восходит к платонизму.
		\item Номинализм: Реально существуют только единичные вещи, а универсалии — это лишь имена (\textit{nomina}), звуки голоса (Росцелин, Уильям Оккам). Принцип Оккама, известный как <<бритва Оккама>>, гласит: <<Не следует умножать сущности без необходимости>>. Этот принцип стал мощным инструментом против усложненных метафизических схем реализма.
		\item Концептуализм (умеренный реализм): Универсалии существуют как понятия (концепты) в человеческом уме, но имеют свое основание в реальных вещах (Пьер Абеляр). Эту же позицию в итоге занял и Фома Аквинский. Он считал, что универсалии существуют трояко: <<до вещей>> (в уме Бога), <<в вещах>> (как их сущность) и <<после вещей>> (в уме человека как результат абстракции).
	\end{itemize}
	
	\subsection{Ключевые представители и идеи}
	\textbf{Ансельм Кентерберийский}
	
	Известен своим онтологическим доказательством бытия Бога: идея Бога как существа, <<совершеннее которого нельзя ничего помыслить>>, необходимо включает в себя и его существование, иначе можно было бы помыслить более совершенное существо (существующее), что является противоречием. Этот аргумент исходит от понятия к бытию (\textit{a priori}).
	
	\textbf{Фома Аквинский (томизм)}
	
	Крупнейший представитель высокой схоластики, систематизировавший христианское учение на основе философии Аристотеля.
	\begin{itemize}
		\item Гармония веры и разума: Истины откровения и истины разума не могут противоречить друг другу, так как их источник один — Бог. Есть истины, доступные только откровению (как учение о Троице), они сверхразумны, но не противоразумны.
		\item Пять доказательств бытия Бога: Рациональные (апостериорные) доказательства, основанные на наблюдении за миром (от движения, от производящей причины, от необходимости и случайности, от степеней совершенства, от целесообразности в природе):
		\begin{enumerate}
			\item Доказательство через движение означает, что всё движущееся когда-либо было приведено в действие чем-то другим, которое в свою очередь было приведено в движение третьим. Именно Бог и оказывается первопричиной всего движения.
			
			\item Доказательство через производящую причину — это доказательство схоже с первым. Так как ничто не может произвести самого себя, то существует нечто, что является первопричиной всего — это Бог.
			
			\item Доказательство через необходимость — каждая вещь имеет возможность как своего потенциального, так и реального бытия. Если мы предположим, что все вещи находятся в потенции, то тогда бы ничего не возникло. Должно быть нечто, что способствовало переводу вещи из потенциального в актуальное состояние. Это нечто — Бог.
			
			\item Доказательство от степеней бытия — люди говорят о различной степени совершенства предмета только через сравнения с самым совершенным. Это значит, что существует самое красивое, самое благородное, самое лучшее — этим является Бог.
			
			\item Доказательство через целевую причину. В мире разумных и неразумных существ наблюдается целесообразность деятельности, а значит, существует разумное существо, которое полагает цель для всего, что есть в мире, — это существо мы именуем Богом.
		\end{enumerate}
		\item Учение о бытии: Различение сущности (\textit{essentia} -- <<что есть вещь>>) и существования (\textit{existentia} -- <<факт бытия вещи>>). Только в Боге они тождественны, во всех сотворенных вещах они различны.
		\item Антропология: Человек — это целостное единство души и тела. Разумная душа, будучи нематериальной, бессмертна.
	\end{itemize}
	
	\section{Западноевропейская философия эпохи Возрождения~\checkmark}
	Философия эпохи Возрождения (XIV–XVI вв.) занимает особое место в истории западной мысли как переходный этап от средневековой схоластики к философии Нового времени. Это время глубокой интеллектуальной и культурной трансформации, вызванной возрождением интереса к античности, формированием новой антропологии, ростом значения науки и искусства, а также появлением новых представлений о человеке, природе и Боге. Возрожденческая философия не представляет собой единого течения — это сложное и противоречивое духовное движение, в котором взаимодействуют гуманизм, неоплатонизм, натурфилософия, скептицизм и зарождающийся рационализм.
	
	Важнейшей отличительной чертой философии эпохи Возрождения оказывается его ориентация на человека. Если в центре внимания древних философов был животворящий Космос, в средние века — Бог, то в эпоху Возрождения — человек.
	
	Возникают и новые философские направления — деизм и панте­изм. Деизм отвергал идею личного Бога и его повседневное вмеша­тельство в жизнь природы и общества. Деизм рассматривал Бога лишь как первопричину, как творца мира, то есть безличное начало, сооб­щившее миру его законы, которые после творения действуют само­стоятельно. Многие из деистов основывали свои представления о ми­ре на новых отраслях естествознания, отстаивали независимость нау­ки от религии. Деизм давал возможность, прикрываясь признанием Бога, рассматривать закономерности природы и общества вне Боже­ственной предопределенности.
	
	В пантеизме Бог и мир отождествлялись. Одним из первых к панте­изму подошел Николай Кузанский. Рассматривая Бога как бесконеч­ный максимум и приближая его к природе как ограниченному макси­муму, он сформулировал идею бесконечности Вселенной. Пантеизм лег в основу большинства натурфилософских учений, противостоящих ре­лигиозному учению о сотворении мира Богом из ничего. Если для древнегреческой фило­софии завершенное и целое — прекраснее незавершенного, то для фи­лософа эпохи Возрождения движение и становление — предпочтитель­нее неподвижно-неизменного бытия. Поэтому в эпоху Возрождения всякая деятельность воспринималась иначе, чем в эпоху раннего средневековья, и даже в эпоху античности. Будучи обусловленной социальной действительностью, фи­лософия активно влияет на общественное бытие, способствует фор­мированию новых идеалов и культурных ценностей. XVII в. открыва­ет следующий период в развитии философии, который принято назы­вать философией Нового времени.
	
	\section{Эмпиризм в философии Нового времени (Ф. Бэкон, Т. Гоббс, Дж. Локк, Дж. Беркли, Д. Юм)~\checkmark}
	
	Эмпиризм — одно из ведущих направлений философии Нового времени, утверждающее, что источником и основанием человеческого знания является чувственный опыт. Эмпиризм противопоставляется рационализму, для которого основой знания служит разум как врождённая способность к выведению истины априори. Развитие эмпиризма в XVII–XVIII веках было связано с формированием науки нового типа и сопровождающим её поворотом к исследованию природы средствами наблюдения и эксперимента. Эмпиризм отражает эту установку и стремится создать философскую эпистемологию, адекватную научной практике.
	
	\par Эмпиризм Нового времени развивался от индуктивной методологии (Бэкон) через материализм и сенсуализм (Гоббс, Локк) к радикальному идеализму (Беркли) и скептицизму (Юм). Каждый следующий автор либо развивает, либо радикализирует тезисы предшественника: от утверждения опыта как основы знания (Бэкон, Локк) — к отрицанию материального мира (Беркли) и недоверию к самому опыту как источнику обоснованных знаний (Юм).
	
	Если Локк сохраняет реалистическое допущение о существовании внешнего мира и разума как средства обобщения, то Беркли разрушает онтологию материального мира, а Юм — эпистемологическое обоснование знания.
	
	\subsection{Фрэнсис Бэкон (1561–1626)}
	Фрэнсис Бэкон считается основателем эмпирической философии и ранней научной методологии. Его главный труд — <<Новый Органон>> — предлагает реформу научного метода на основе индукции. В противоположность силлогистическому методу Аристотеля, основанному на дедукции, Бэкон утверждает, что знание должно быть выведено из конкретных наблюдаемых фактов посредством постепенного обобщения. Он вводит понятие <<идолов>> (призраков), то есть ложных представлений, мешающих объективному познанию: идолы рода, пещеры, рынка и театра.
	
	Бэкон аргументирует, что истинное знание — это знание, способное управлять природой, и потому научная философия должна ориентироваться на практическую полезность. Его подход противопоставлен средневековой схоластике, опирающейся на авторитет и умозрение.
	
	Контраргументы против методологического оптимизма Бэкона заключаются в том, что индукция сама по себе не гарантирует истинности знаний, если она не сопровождается теоретической интерпретацией. Этот критический пункт впоследствии будет развит Юмом в его <<проблеме индукции>>.
	
	\subsection{Томас Гоббс (1588–1679)}
	Гоббс, хотя и считается политическим философом, был также важной фигурой в эмпирической эпистемологии. В онтологии он последовательно материалистичен: всё существующее есть тело, а мышление — это движение в мозге, вызванное воздействием внешних объектов. Его гносеология основана на том, что все знания восходят к ощущениям, которые преобразуются в память и воображение, а затем — в рассуждение.
	
	Он утверждает, что разум — это не самостоятельный источник знаний, а только инструмент упорядочивания чувственного материала. Он также отказывается от идеи врождённых понятий.
	
	Политическая философия Гоббса в <<Левиафане>> также имеет эмпирическое основание: государство рассматривается как продукт человеческих желаний, страха и стремления к самосохранению, выявленных через наблюдение за природой человека.
	
	Критики указывают на редукционизм Гоббса, который сводит сознание и мышление к механическому движению, и на недостаток внимания к качественной стороне субъективного опыта.
	
	\subsection{Джон Локк (1632–1704)}
	
	Локк — ключевая фигура в становлении классического эмпиризма. В <<Опытe о человеческом разумении>> он утверждает, что человеческий ум при рождении — это tabula rasa (чистая доска), на которой опыт вписывает все идеи. Он разграничивает два типа опыта: внешний (ощущения) и внутренний (рефлексия). Все идеи происходят либо из ощущений, либо из рефлексии над ними.
	
	Он различает простые идеи, которые воспринимаются непосредственно, и сложные, которые формируются из простых посредством ассоциаций и операций разума. Также вводит различие между первичными (неотъемлемыми от объекта: протяжённость, движение) и вторичными качествами (цвет, вкус), которые существуют только в восприятии субъекта.
	
	Аргументация Локка направлена против концепции врождённых идей (например, у Декарта), которую он считает недоказуемой. Однако критики, такие как Лейбниц, утверждают, что в чистом опыте невозможно объяснить наличие общих и необходимых истин, как, например, в математике.
	
	\subsection{Джордж Беркли (1685–1753)}
	
	Беркли радикализирует эмпиризм, выступая против материалистической интерпретации внешнего мира. В своём основном труде <<Трактат о началах человеческого знания>> он утверждает, что <<быть — значит быть воспринимаемым>> (esse est percipi). Согласно Беркли, вся реальность — это совокупность восприятий, а материя как субстанция вне сознания — ненужная гипотеза.
	
	Он различает идеи, которые мы воспринимаем, и дух (духовную субстанцию), который воспринимает. Бог, по Беркли, есть универсальный дух, который обеспечивает постоянство и согласованность восприятий.
	
	Таким образом, он отрицает существование материи, но сохраняет объективный мир, полагая, что он существует в восприятии Бога. Это делает его философию имматериалистическим идеализмом.
	
	Критики Беркли, включая Юма, указывают на трудность объяснения постоянства мира без материальной субстанции, а также на то, что его аргументация подрывает доверие к объективной науке.
	
	\subsection{Давид Юм (1711–1776)}
	Юм завершает линию классического британского эмпиризма, доводя её до скептицизма. В <<Трактате о человеческой природе>> и <<Исследовании о человеческом разумении>> он утверждает, что все элементы сознания сводятся к впечатлениям (непосредственным чувственным данным) и идеям (бледным копиям впечатлений).
	
	Юм радикализирует проблему причинности: он показывает, что представление о причинной связи невозможно вывести из опыта, так как мы никогда не воспринимаем необходимую связь между событиями — только их регулярную последовательность. Отсюда вытекает <<проблема индукции>>: из того, что два события регулярно следуют одно за другим, не следует, что так будет всегда.
	
	Он также отрицает существование устойчивого <<я>> как субстанции — по его мнению, сознание представляет собой поток восприятий без постоянного носителя.
	
	Юм приходит к скептицизму: разум не способен обосновать ни причинность, ни существование внешнего мира, ни даже сам субъект — но в обыденной жизни мы, следуя привычке, всё же продолжаем действовать, как если бы всё это существовало.
	
	Кант в <<Критике чистого разума>> признаёт, что именно Юм <<разбудил его от догматической дремоты>> и побудил к созданию трансцендентальной философии, сочетающей элементы эмпиризма и рационализма.
	
	\section{Рационализм в философии Нового времени (Р. Декарт, Б. Спиноза, Г.В. Лейбниц)~\checkmark}
	Рационализм — философское направление, согласно которому разум является основным источником и критерием истинного знания. В отличие от эмпиризма, который делает ставку на чувственный опыт, рационализм утверждает, что знание возможно априори, т.е. до и независимо от опыта, а чувственные данные ненадёжны и нуждаются в рациональной проверке. Классический рационализм формируется в XVII веке как философская основа новой науки, стремящейся к универсальной, математически строгой картине мира.
	
	Классический рационализм развивается от дуализма Декарта, через монизм Спинозы, к плюрализму Лейбница. Каждый философ предлагает свой способ построения всеобъемлющей рациональной системы:
	\begin{enumerate}
		\item Декарт ищет достоверное основание знания в разуме и ясных идеях, но сохраняет дуализм;
		
		\item Спиноза отказывается от дуализма в пользу строгого монизма и рационального детерминизма;
		
		\item Лейбниц стремится к согласованию свободы, индивидуальности и универсального порядка через концепцию монад и предустановленной гармонии.
	\end{enumerate}
	
	Общим для всех является убеждённость в способности разума постичь структуру мира независимо от чувственного опыта. Они признают существование врождённых идей и рассматривают философию как системное знание, аналогичное математике.
	
	Противоположной линией развития является эмпиризм, для которого разум — лишь инструмент обработки данных, полученных из опыта. Однако после Юма и Канта философия сделала попытку синтеза этих направлений: трансцендентальная философия Канта стремится соединить рационалистскую и эмпиристскую традиции, показывая, как априорные формы сознания структурируют опыт.
	
	\subsection{Рене Декарт (1596–1650)}	
	Декарт считается основоположником современного рационализма. Его методологический скептицизм представлен в <<Рассуждении о методе>> и <<Метафизических размышлениях>>. Начав с сомнения во всём, что может быть подвергнуто сомнению, Декарт приходит к несомненной истине: cogito, ergo sum — <<мыслю, следовательно, существую>>. Это положение становится первым достоверным знанием, на котором строится вся дальнейшая система.
	
	Декарт вводит идею врождённых идей (например, идея Бога, бесконечности, числа), которые не могут быть получены из опыта. Он различает res cogitans (мыслящую субстанцию) и res extensa (протяжённую субстанцию), что приводит к дуализму: человек — это соединение души (разума) и тела (механического организма). Мир, по Декарту, устроен как механизм и подчиняется математическим законам.
	
	Аргументация Декарта: только разум способен преодолеть иллюзии чувств и привести к достоверному знанию. Для обоснования истины он вводит критерий ясности и отчётливости идей.
	
	Контраргументация касается дуализма: как две субстанции с различной природой (мышление и протяжённость) могут взаимодействовать? Эта <<проблема взаимодействия>> станет объектом критики как со стороны эмпириков (Гоббс), так и позднейших рационалистов.
	
	\subsection{Бенедикт Спиноза (1632–1677)}
	
	Спиноза радикализует рационализм и устраняет дуализм Декарта. В <<Этике>>, написанной в геометрической форме, он утверждает, что существует только одна субстанция — Бог или природа (Deus sive Natura), которая содержит в себе всё сущее. Это философский монизм: разум, тело, природа, Бог — это нераздельные проявления одной и той же субстанции, различающиеся только по атрибутам (мышление и протяжённость).
	
	Человек, по Спинозе, — это модус (модификация) субстанции, и его разум, как часть природы, способен постигать её рационально. Истинное знание достигается через интуитивное и рациональное постижение необходимой связи вещей. Эмоции объясняются как состояния, проистекающие из ограниченного понимания причин.
	
	Спиноза отрицает свободную волю: всё в мире детерминировано. Однако свобода в его системе — это познание необходимости и согласие с ней. Через разум человек постигает структуру мира и достигает amor Dei intellectualis — интеллектуальной любви к Богу.
	
	Контраргументы: система Спинозы упрекается в фатализме и невозможности объяснения индивидуальности и субъективного опыта. Кроме того, её сложность и отчуждённость от эмпирической науки вызывали скепсис у последующих философов.
	
	\subsection{Готфрид Вильгельм Лейбниц (1646–1716)}
	Лейбниц стремится примирить рационализм с идеей индивидуальности и гармонии мира. В отличие от монизма Спинозы, он вводит концепцию множественности субстанций — монад. Монада — это неделимая, нематериальная, духовная единица, отражающая в себе весь космос с позиции своей уникальной точки зрения. Монады не взаимодействуют между собой причинно, но находятся в состоянии предустановленной гармонии, установленной Богом.
	
	Лейбниц различает истины разума (необходимые, априорные, подчинённые принципу тождества и непротиворечия) и истины факта (контингентные, познаваемые через опыт, но подчинённые принципу достаточного основания). Он вводит принцип предустановленной гармонии: все монады развиваются согласно внутренней программе, как часы, синхронизированные Богом.
	
	Одно из важных понятий Лейбница — возможные миры. Бог, как высшее существо, выбирает из всех возможных миров тот, который обладает наибольшей полнотой и гармонией. Отсюда тезис: <<Мы живём в лучшем из возможных миров>>.
	
	Лейбниц также защищает наличие врождённых идей, но не в форме готовых понятий, а как склонностей или способностей ума.
	
	Контраргументация: критики (например, Вольтер в <<Кандиде>>) высмеивали оптимизм Лейбница, считая его философию оторванной от реальной жизни, полной страданий и зла. Также вызывает сомнение непроверяемость метафизических понятий, таких как монады и гармония.
	
	\section{Западноевропейская философия Просвещения~\checkmark}
	
	XVII и XVIII века - это время особых исторических изменений в странах Западной Европы. В этот период наблюдается становление и развитие промышленного производства. Все активнее осваиваются в чисто производственных целях новые природные силы и явления: строятся водяные мельницы, конструируются новые подъемные машины для шахт и т.д. Все эти и другие инженерные работы выявляют очевидную потребность общества в развитии конкретно-научного знания. Уже в XVII веке многие полагают, что <<знание - сила>> (Ф.Бэкон), что именно <<практическая философия>> (конкретно-научное знание) поможет нам с пользой для нас овладеть природой и стать <<господами и хозяевами>> этой природы (Р.Декарт).
	
	В XVIII веке еще более закрепляется безграничная вера в науку, в наш разум. Для многих мыслителей XVIII века научный прогресс начинает выступать как необходимое условие успешного продвижения общества по пути к человеческой свободе, к счастью людей, к общественному благополучию. При этом принималось, что все наши действия, все поступки лишь тогда могут быть гарантированно успешными, когда они будут пронизаны светом знаний, будут опираться на достижения наук. Поэтому главной задачей цивилизованного общества объявлялось всеобщее просвещение людей.
	
	Многие мыслители XVIII века уверенно стали объявлять, что первой и главной обязанностью любого <<истинного друга прогресса и человечества>> является <<просветление умов>>, просвещение людей, приобщение их ко всем важнейшим достижениям науки и искусства. Эта установка на просвещение масс стала настолько характерной для культурной жизни европейских стран в XVIII веке, что впоследствии XVIII век был назван веком Просвещения, или эпохой Просвещения.
	
	Первой в эту эпоху вступает Англия. Для английских просветителей (Д. Локк, Д. Толанд, М. Тиндаль и др.) была характерна борьба с традиционным религиозным мировосприятием, которое объективно сдерживало свободное развитие наук о природе, о человеке и обществе. Идейной формой свободомыслия в Европе с первых десятилетий XVIII века становится деизм. Деизм не отвергает бога как творца всей живой и неживой природы, но в рамках деизма жестоко постулируется, что это творение мира уже свершилось, что после этого акта творения бог не вмешивается в природу: теперь природа ничем внешним не определяется и теперь причины и объяснения всех событий и процессов в ней следует искать только в ней самой, в ее собственных закономерностях.
	
	Английское просвещение было просвещением для избранных, носило аристократический характер. В отличие от него французское просвещение ориентировано не на аристократическую элиту, а на широкие круги городского общества. Именно во Франции в русле этого демократического просвещения зарождается идея создания <<Энциклопедии, или толкового словаря наук, искусств и ремесел>>, энциклопедии, которая бы в простой и доходчивой форме (а не в форме научных трактатов) знакомила читателей с важнейшими достижениями наук, искусств и ремесел.
	
	Идейным вождем этого начинания выступает Д. Дидро, а его ближайшим соратником - Д. Аламбер. По замыслу Д. Дидро в <<Энциклопедии>> должны были отражаться не только достижения конкретных наук, но и многие новые философские концепции относительно природы материи, сознания, познания и т.д. в <<Энциклопедии>> стали помещаться статьи, в которых давались критические оценки традиционной религиозной догматики, традиционного религиозного мировосприятия. Все это определило негативную реакцию церковной элиты и определенного круга высших государственных чиновников к изданию <<Энциклопедии>>.
	
	В Германии движение Просвещения связано с деятельностью Х. Вольфа, И. Гердера, Г. Лессинга и др. Если иметь в виду популяризацию наук и распространение знаний, то здесь особую роль играет деятельность Х. Вольфа. Его заслуги отмечали впоследствии и И.Кант, и Гегель.
	
	Философия для Х. Вольфа - это <<мировая мудрость>>, предполагающая научное объяснение мира и построение системы знаний о нем. Он доказывал практическую полезность научных знаний. Сам он известен был и как физик, и как математик, и как философ. И характеризуется он часто как отец систематического изложения философии в Германии (И. Кант). Работы свои писал Х. Вольф на простом и доходчивом языке. Его философская система излагалась в учебниках, заменивших схоластические средневековые курсы во многих странах Европы (в том числе и в Киеве, а затем и в Москве). Х. Вольф был избран членом многих академий Европы.
	
	У Х. Вольфа учились М.В. Ломоносов, Ф. Прокопович и другие наши соотечественники, проходившие учебу в Германии. Он не отвергал бога как творца мира, и ту целесообразность, которая характерна для природы, для всех ее представителей, он связывал с мудростью бога: при сотворении мира бог все продумал и все предусмотрел. Но утверждая простор для развития естественных наук, Х. Вольф оставался сторонником деизма, что несомненно предопределило в последующем деизм М.В. Ломоносова.
	
	Подводя итоги сказанному выше о философии Просвещения, можно отметить следующие важные моменты в ее общей характеристике:
	
	\begin{enumerate}
		\item получает заметное развитие глубокая вера в неограниченные возможности науки в познании мира - вера, в основании которой лежали хорошо усвоенные философами Просвещения идеи Ф. Бэкона (о возможностях опытного исследования природы) и Р. Декарта (о возможностях математики в естественнонаучном познании);
		
		\item развиваются деистические представления о мире, что в свою очередь приводит к формированию материализма как достаточно цельного философского учения, именно деизм в единстве с успехами и результатами естественных наук приводит в результате к формированию французского материализма XVIII века;
		
		\item формируется новое представление об общественной истории, о ее глубокой связи с достижениями науки и техники, с научными открытиями и изобретениями, с просвещением масс.
	\end{enumerate}
	
	\section{Теория познания и этика И. Канта~\checkmark}
	Родоначальником немецкой классической философии является И. Кант (1724–1804 гг.). В докритический период своего творчества до 1770 г., И. Кант выступал, как крупный ученый – астроном, физик, географ. В работах этого периода таких как, <<Всеобщая естественная история и теория неба>> (1755 г.) И. Кант выступал, как стихийный материалист и диалектик, обосновывающий идею саморазвития природы.
	
	В критический период Кант создает три основные работы, которые и дали название данному периоду:
	\begin{enumerate}
		\item 	<<Критика чистого разума>> (1781 г.);
		\item 	<<Критика практического разума>> (1788 г.);
		\item 	<<Критика способности суждения>> (1790 г.).
	\end{enumerate}
	И. Кант в этих трех работах связанных единым замыслом обосновывающим систему <<трансцендентального идеализма>>(так называлась философия Канта), осуществил коренной переворот в постановке и решении проблем философии. В средневековой философии и философии эпохи Возрождения центральной частью философских систем является учение о бытии – онтология. Философы Нового времени – Спиноза, Локк, Бэкон, Декарт, Беркли, Юм перенесли акцент на проблемыгносеологии. Однако в центральной проблеме гносеологии – взаимодействие субъекта и объекта – докантовская философия акцентировала внимание на анализе объекта познания. И. Кант делает предметом философии специфику познающего субъекта, его способность к познанию, которая, по его мнению, определяет способ познания и контролирует предмет знания. Таким образом, Кант совершает <<коперниканский переворот>> в философии. Эту задачу Кант решает в своей работе <<Критика чистого разума>> (1781 г.).
	
	Согласно Канту, у человека есть три средства познания: чувства, рассудок (предметное мышление), разум (мышление о мышлении).
	Трем видам познания соответствуют: математика, естествознание, философия.
	
	Чувственное познание (арифметика, геометрия) возможны в силу того, что нам априорно присущи пространство и время. <<Я называю трансцендентальным, – пишет Кант, всякое познание, занимающееся не только предметом, сколько видами нашего познания предметов, поскольку это познание должно быть возможным a priori>>.
	
	Но чувственное познание дает нам лишь представление о предмете, явлении, лишь упорядоченный априорными формами (пространством и временем) эмпирический материал, который должен быть осмыслен рассудком.
	
	<<Рассудок ничего не может созерцать, а чувства ничего не могут мыслить. Только из их соединения может возникнуть знание>>.
	
	Рассудок подводит многообразие впечатлений чувственного опыта под единство понятия. Эта способность рассудка обусловлена тем, что ему a priori присущи категории – всеобщие понятия, выступающие формами мышления.
	
	Кант выделяет четыре группы категорий: количество, качество, отношения, модальность.
	
	Рассудок не пассивно осмысливает данные чувственного опыта, а конструирует образ предмета с помощью категорий. Но в силу этого образа предмета, имеющийся в нашем сознании, и сам предмет не тождественны. Вещь, существующая в нашей голове, – это явление (феномен), а вещь, существующая сама по себе, – это вещь в себе>> (ноумен), и они в принципе не могут совпадать, так как мы не можем познавать объективный мир, не накладывая на него свои познавательные формы, не конструируя его. Кант ограничивает возможности познания только миром явлений (феноменов) и отрывает, таким образом явление от сущности. По сути Кант выступает, как непоследовательный агностик, в отличие от полного агностицизма Д. Юма.
	
	По мнению И. Канта, задача разума иная нежели задача чувственности и рассудка. Разум выходит за пределы чувственности, опыта, он формирует не трансцендентальные основоположения, а трансцендентальные принципы (за пределами опыта), дающие возможность осуществить высшее единство знания.
	
	Разум – это высшая способность субъекта, которая руководит деятельностью рассудка, ставит перед ним цели. Разум оперирует идеями. Доказательству положения о том, что разум опирается на мнимые идеи, служит учение Канта об антиномиях разума.
	
	Антиномии – это противоречивые взаимоисключающие положения (мир конечен – мир бесконечен, Бог есть – Бога нет и т. д.).
	
	Таким образом, философия, как система особого философского знания невозможна. Но она возможна, во-первых, как критическая философия, исследующая наши способности к познанию; во-вторых, философия необходима для того, чтобы исследовать конечную цель существования человека, то есть как мораль.
	
	В этике, как и в познании, Кант пытается найти всеобщее правило морали, не вытекающее из опыта. Он исходит из того, что долженствование – <<ты должен>> – есть абсолютное обязательство, не вытекающее из опыта. Если это так, то должен быть всеобщий абсолютный принцип, согласно которому человек должен действовать, всеобщее моральное правило. Кант в <<Критике практического разума>> формирует этот принцип, называя его категорическим императивом. Он формирует его следующим образом: <<поступай так, чтобы максима твоей воли могла в то же время иметь силу принципа всеобщего законодательства>>. Это значит следующее: поступай по отношению к другим так, как ты хотел бы, чтобы они поступали по отношению к тебе.
	
	В политической теории Кант исходит, прежде всего, из прав индивида, выводя их из трансцендентальных свойств человека. Если в морали человек должен относиться к другому человеку, как к цели, то и в праве, он должен исходить из уважения прав других. Кант был одним из первых, кто заговорил о правах человека и правовом государстве.
	
	\section{Абсолютный идеализм Г.В.Ф. Гегеля~\checkmark}
	Георг Вильгельм Фридрих Гегель (1770–1831) является одним из ключевых представителей немецкой классической философии и основателем системы философского идеализма, известной как абсолютный идеализм. Основной задачей Гегеля было объяснение того, как философия может прояснить структуру и развитие реальности, одновременно обеспечив методологическую цельность и внутреннюю согласованность своего учения.
	
	Абсолютный идеализм Гегеля основывается на понимании реальности как логического процесса, в котором развитие мысли и бытия составляет единую, органичную, самодвигающуюся и самосозидательную систему. Ключевые аспекты его философии можно выделить следующим образом:
	
	\begin{enumerate}
		\item Идея как основа реальности. В отличие от классического эмпиризма или материализма, которые исходят из того, что реальность состоит из конкретных объектов, Гегель утверждает, что реальность представляет собой проявление Абсолютной Идеи. В его философии идея — это не абстракция, а нечто живое, динамическое, что воплощается в процессе истории, культуры и природы.
		
		\item Диалектический метод. Гегель использует диалектику как метод познания, который раскрывает развитие через противоречия. Согласно этому методу, каждая концепция, каждая форма бытия и мышления (тезис) сталкивается с противоположной концепцией (антитезис), что приводит к их разрешению в новой, более высокоразвивающейся форме (синтез). Этот процесс бесконечен и сопровождается развитием как мыслительных категорий, так и всей реальности.
		
		\item Развитие через противоречие. Важным элементом диалектики является то, что каждое противоречие, каждое столкновение противоположностей не уничтожает реальность, а способствует её более глубокой и полной реализации. Противоречия внутри Абсолютной Идеи необходимы для её самопознания и реализации.
		
		\item Абсолютный дух. Для Гегеля процесс самореализации Идеи является процессом, в котором все явления в мире связаны и развиваются к абсолютному знанию. Абсолютный дух включает в себя как субъективное, так и объективное сознание, а также абсолютное знание. В конечном счете, абсолютный дух реализуется в истории человечества, в искусстве, религии и философии, которые выражают различные уровни самопонимания.
		
		\item Исторический процесс. Гегель рассматривает историю как процесс, в котором разум или дух постепенно проявляют себя и достигают своей полной реализации. Это развитие включает в себя как прогресс в сознании человека, так и прогресс в политических институтах и моральных ценностях. История для Гегеля — это не случайный набор событий, а рационально устроенный процесс, в котором каждое событие имеет своё место и значение.
	\end{enumerate}
	
	Абсолютный идеализм Гегеля в значительной степени сформировался в контексте философии, развитой до него. В его учении можно выделить несколько ключевых влияний:
	
	\begin{enumerate}
		\item Кантианство. Гегель был глубоко знаком с философией Иммануила Канта, который утверждал, что мы не можем познать вещи "в себе", а только феномены, то есть явления, которые представляют собой формы нашего восприятия. Гегель, в свою очередь, критиковал Канта за его ограниченность, утверждая, что разум может познать реальность в её абсолютном выражении, а не только в формах нашего восприятия.
		
		\item Фихте и Шеллинг. В философии Иоганна Готлиба Фихте идея становилась центральной, и философия была направлена на развитие субъекта. Гегель унаследовал этот акцент на идею, но в отличие от Фихте, который фокусировался на субъективном аспекте, Гегель утверждал, что разум и идея существуют в объективной реальности, их развитие происходит не только через индивидуальное сознание, но и через историю.
		
		\item Марксизм. Впоследствии философия Гегеля оказала влияние на Карла Маркса. Хотя Маркс в своей философии отверг абсолютный идеализм и разработал материалистическую диалектику, его понимание истории и социальных изменений во многом заимствовано из гегелевской диалектики. В отличие от Гегеля, Маркс подчеркивал роль материальных условий и классовой борьбы в развитии общества, в то время как для Гегеля основным двигателем является идея.
	\end{enumerate}
	
	Аргументация Гегеля основывается на идее, что весь мир, вся реальность представляет собой единую систему, в которой каждое событие, каждый элемент находится в развитии и направлен к своему завершению в Абсолютной Идее. Этот процесс можно понимать как синтез разумного и действительного. Ключевым моментом является также то, что для Гегеля все явления имеют свою внутреннюю необходимость и оправдание.
	
	Однако контраргументы против философии Гегеля часто касаются его восприятия истории как предопределенного процесса, где каждый момент имеет свою роль в более высоком смысле. Это ставит под сомнение свободу воли человека и возможность реальных изменений в истории, помимо логического процесса. Кроме того, критики утверждают, что Гегель слишком идеализирует концепцию Абсолютной Идеи, лишая её практического аспекта.
	
	\section{Философия К. Маркса}
	
	\section{Философия Ф. Ницше}
	
	\section{Философия позитивизма в XIX веке: основные представители и идеи}
	
	\section{Русская религиозная философия XIX–XX вв. (Ф.М. Достоевский, Л.Н. Толстой, В.С. Соловьев, Н.А. Бердяев)~\checkmark}
	
	Самобытное направление русской мысли, для которого характерны тесная связь с православной духовностью, литературой и острое внимание к общественным проблемам. Важнейшей предпосылкой ее развития стал спор западников (П.Я. Чаадаев, А.И. Герцен), считавших, что Россия должна идти по европейскому пути развития, и славянофилов (А.С. Хомяков, И.В. Киреевский), которые отстаивали идею особого пути России, основанного на православии и общинности. Славянофилы ввели ключевое понятие соборности — свободного духовного единения людей в Церкви, основанного на любви, в противоположность западному индивидуализму. Ключевые темы русской философии: критика западного рационализма и индивидуализма, идея соборности, смысл истории и эсхатологические мотивы.
	
	\subsection{Ф.М. Достоевский}
	В своем творчестве ставил глубочайшие философские вопросы. Центральная тема — трагическая природа свободы, которая может вести как к богочеловечеству, так и к своеволию и человекобожеству (<<Легенда о Великом инквизиторе>>). Его формула <<Если Бога нет, то все позволено>> исследует основы морали, показывая, что без веры в бессмертие души нравственность теряет свой абсолютный фундамент.
	
	\subsection{Л.Н. Толстой}
	Развил учение о \textit{непротивлении злу насилием}, основанное на радикальном переосмыслении Евангелия. Критиковал государственные и церковные институты как основанные на насилии и искажающие учение Христа. Смысл жизни видел в нравственном самоусовершенствовании, опрощении и служении людям.
	
	\subsection{В.С. Соловьев}
	Основоположник систематической русской религиозной философии. Центральная идея его учения — \textit{философия всеединства}, согласно которой мир есть органическое целое, единство Бога и творения. Разрабатывал учение о \textit{Софии, Премудрости Божией}, как душе мира и идеальном человечестве. Цель истории, по Соловьеву, — достижение \textit{Богочеловечества}, свободного единения человечества с Богом, в котором божественное и человеческое начала взаимно проникают друг в друга, не сливаясь и не разделяясь.
	
	\subsection{Н.А. Бердяев}
	Представитель русского экзистенциализма.
	В центр своей философии ставил проблему свободы, которая, по его мнению, первичнее бытия и коренится не в Боге, а в <<безосновной основе>> (\textit{Ungrund}). Именно эта до-божественная свобода является источником как мирового зла, так и человеческого \textit{творчества}. Бог не творил свободу, а потому не всесилен над ней и не ответственен за зло, проистекающее из нее. Смысл человеческой жизни видел в творчестве, в творческом ответе человека на Божий призыв к со-творению мира. Резко критиковал <<мир объективации>> — отчужденный, безличный мир природы и социума, где свобода умирает и личность заменяется индивидом.
	
	\section{Философия психоанализа: основные представители и идеи~\checkmark}
	
	Психоанализ — это не только метод терапии, но и философская концепция, раскрывшая роль бессознательного в жизни человека и культуре.
	
	\subsection{Зигмунд Фрейд (классический психоанализ)}
	\begin{itemize}
		\item Предложил структурную модель психики: \textit{Оно} ($Id$ -- бессознательные влечения, принцип удовольствия), \textit{Я} ($Ego$ -- сознание, принцип реальности, посредник между Оно и Сверх-Я) и \textit{Сверх-Я} ($Super-Ego$ -- совесть, социальные нормы, интериоризированный образ отца).
		\item Движущими силами психики считал влечения жизни (\textbf{Эрос}, включая либидо — сексуальную энергию) и смерти (\textbf{Танатос}, влечение к разрушению). Ключевым для развития личности является \textbf{Эдипов комплекс} — бессознательное влечение ребенка к родителю противоположного пола и ревность к родителю своего пола. Преодоление этого комплекса через идентификацию с родителем своего пола является решающим шагом в формировании <<Сверх-Я>>.
		\item Культуру рассматривал как результат \textit{сублимации} (переключения) сексуальной энергии на несоциальные цели (творчество, наука), основанный на подавлении и вытеснении первичных влечений. Культура, таким образом, имеет невротическую природу и оплачивается ценой <<недовольства>>.
	\end{itemize}
	
	\subsection{Карл Густав Юнг (аналитическая психология)}
	\begin{itemize}
		\item Расширил понятие бессознательного, введя концепцию \textit{коллективного бессознательного}, общего для всего человечества и содержащего опыт предков. В отличие от фрейдовского либидо как сексуальной энергии, Юнг понимал \textbf{либидо} как общую психическую энергию.
		\item Оно состоит из \textit{архетипов} — универсальных врожденных психических структур. Ключевые архетипы: \textbf{Тень} (темная, вытесненная сторона личности), \textbf{Анима/Анимус} (внутренний образ женщины в мужчине и мужчины в женщине), \textbf{Самость} (архетип целостности, центр личности).
		\item Целью психического развития Юнг считал \textit{индивидуацию} — процесс становления личности, достижение целостности через сознательную интеграцию личного и коллективного бессознательного, в первую очередь — через диалог с Тенью и другими архетипами.
	\end{itemize}
	
	\subsection{Неофрейдизм и постфрейдизм}
	\begin{itemize}
		\item Альфред Адлер считал главной движущей силой не либидо, а врожденное \textbf{чувство неполноценности} и стремление к его компенсации через достижение превосходства, власти и социального признания.
		\item Эрих Фромм соединял психоанализ с марксизмом, анализируя социальные причины отчуждения и \textit{бегства от свободы}. Он утверждал, что капитализм порождает тип человека, который боится свободы и ищет спасения в подчинении авторитарному лидеру или в конформизме.
		\item Жак Лакан переосмыслил психоанализ через структурализм, выдвинув тезис <<бессознательное структурировано как язык>>. Он выделил три регистра психического: \textbf{Воображаемое} (стадия зеркала, формирование целостного, но отчужденного образа <<Я>>), \textbf{Символическое} (мир языка, закона, культуры, <<Имя-Отца>>, который вводится через эдипов комплекс) и \textbf{Реальное} (невыразимая, травматическая реальность, лежащая за пределами символизации).
	\end{itemize}
	
	\section{Феноменология: основные представители и идеи}
	
	\section{Экзистенциализм: основные представители и идеи~\checkmark}
	Экзистенциализм — это направление в европейской философии XX века, в центре которого стоит человек как существующее (экзистирующее) существо, переживающее тревогу, свободу, выбор, ответственность, абсурд и конечность. В отличие от системной и рационалистической философии Нового времени, экзистенциализм фокусируется не на сущности или природе человека, а на его существовании — как уникальном, личном, конкретном и неповторимом опыте бытия.
	
	Экзистенциализм имеет глубокие корни. Его предшественниками можно считать Сократа, который утверждал, что философия начинается с заботы о собственной душе и самопознания, Августина, подчёркивавшего внутреннюю драму человеческого выбора и напряжение между свободой и благодатью, Блеза Паскаля, который говорил о человеке как о <<мыслящем тростнике>>, уязвимом и обречённом на сомнение.
	
	Однако непосредственными основоположниками экзистенциальной философии считаются Сёрен Кьеркегор и Фридрих Ницше, а её основное развитие приходится на середину XX века в трудах Мартена Хайдеггера, Жана-Поля Сартра, Альбера Камю, Карла Ясперса, Габриэля Марселя и других.
	
	\subsection{Основные идеи экзистенциализма}
	\begin{enumerate}
		\item Первичность существования над сущностью (existence precedes essence): у человека нет заранее заданной сущности; он сам становится тем, кем он себя делает через свободный выбор (Сартр).
		
		\item Радикальная свобода и ответственность: свобода — это не просто право выбора, но и бремя, за которое человек полностью отвечает. Невозможно переложить вину или долг на Бога, природу, общество или историю.
		
		\item Абсурд и тревога: человек живёт в мире, не предоставляющем ему готовых смыслов. Столкновение между стремлением к смыслу и молчанием Вселенной рождает абсурд (Камю), а осознание ничтожности и неопределённости бытия — экзистенциальную тревогу.
		
		\item Смерть и конечность: важнейшее условие человеческой свободы — признание конечности. Осознание смерти разрушает иллюзии повседневности и делает выбор подлинным (Хайдеггер).
		
		\item Аутентичность (подлинность): быть самим собой, а не тем, кем требует быть общество (<<люди>>, <<они>>), — это основной экзистенциальный императив.
		
		\item Отчуждение, одиночество, брошенность: человек осознаёт, что он брошен в мир, где он никому не гарантирован и не защищён. Он должен сам придать смысл своей жизни.
	\end{enumerate}
	
	\subsection{Основные представители и их философские позиции}
	\begin{enumerate}
		\item Сёрен Кьеркегор (1813–1855) — христианский экзистенциализм
		
		Кьеркегор вводит понятие экзистенции как субъективного, личного способа бытия. Он противопоставляет эстетическую, этическую и религиозную стадии существования. Главная тема — прыжок веры, который невозможен рационально, но необходим для выхода из отчаяния. Кьеркегор считал, что подлинная жизнь возможна лишь в религиозном опыте, а не в системах Гегеля и других рационалистов.
		
		\item Фридрих Ницше (1844–1900) — антирелигиозный экзистенциализм
		
		Ницше утверждает, что <<Бог умер>>, и этим подрывает всю прежнюю метафизику. Он вводит понятие воли к власти, сверхчеловека, вечного возвращения. Его философия призывает к утверждению жизни во всей её жестокости, случайности и иррациональности. Человек должен сам стать творцом ценностей, освободившись от морали рабов.
		
		\item Мартин Хайдеггер (1889–1976) — онтологический экзистенциализм
		
		В работе <<Бытие и время>> Хайдеггер вводит ключевое понятие Dasein — бытие-человека-в-мире. Он показывает, что человек — это не сущность, а проект, устремлённый в будущее. Главное состояние Dasein — заброшенность и бытие-к-смерти. Через тревогу и смерть человек выходит к подлинности, отрываясь от повседневной безличной <<болтовни>> и <<людей>>.
		
		\item Жан-Поль Сартр (1905–1980) — атеистический экзистенциализм
		
		Сартр утверждает: <<человек осуждён быть свободным>>, потому что Бога нет, и, следовательно, нет и заданной человеческой природы. Его философия — это этика свободы, основанная на индивидуальной ответственности. Он противопоставляет подлинное существование — принятие свободы — дурной вере — бегству от свободы, самообману, попытке <<быть как все>>.
		
		\item Альбер Камю (1913–1960) — экзистенциализм абсурда
		
		Камю, особенно в <<Мифе о Сизифе>> и <<Чуме>>, рассматривает абсурд как главный опыт современного человека: мир не имеет смысла, и человек должен жить, несмотря на это. Он предлагает бунт без иллюзий — отказ от религии и самоубийства, но и от нигилизма. Герой Камю — это Сизиф, обречённый, но непокорный, находящий смысл в самом действии.
		
		\item Карл Ясперс (1883–1969) — философия предельных ситуаций
		
		Для Ясперса экзистенция раскрывается через предельные ситуации — смерть, страдание, вина, борьба. Эти моменты вскрывают ограниченность рационального знания и ведут к экзистенциальному просветлению. Он сохраняет христианскую религиозность, но делает акцент на внутреннем переживании и свободе веры.
	\end{enumerate}
	
	Экзистенциализм утверждает:
	\begin{enumerate}
		\item что человек — не объект, и философия должна исходить из его живого, конкретного бытия;
		
		\item что жизнь — это проект, а не исполнение предписанных норм;
		
		\item что свобода — это онтологическая структура человека, а не просто социальная возможность.
	\end{enumerate}
	
	Критики экзистенциализма, особенно из лагеря аналитической философии, упрекали его в ненаучности, смутности, литературности и психологизме. Например, Айер и Рассел полагали, что экзистенциализм не даёт проверяемых аргументов и уходит в поэтические метафоры. Кроме того, экзистенциализм иногда критиковался за пессимизм, индивидуализм и неспособность предложить политическое решение социальных проблем.
	
	Однако защитники (в том числе Хабермас) указывали, что экзистенциализм возрождает этическое измерение философии, персональное участие и ответственность, возвращает философии экзистенциальную глубину и антропологическую честность.
	
	\section{Аналитическая философия: основные представители и идеи~\checkmark}
	\par Аналитическая философия — это одно из крупнейших и наиболее влиятельных направлений в философии XX века, которое отличается стремлением к ясности, точности и логической строгости. В отличие от традиционных метафизических школ, аналитическая философия фокусируется на языке, логике и значении понятий. Развитие аналитической философии тесно связано с появлением философской логики, философии языка и философии науки.
	
	Аналитическая философия начинается с представления, что философия должна быть научной и логически строгой дисциплиной, которая должна опираться на анализ языка и логических структур. Одним из основных принципов является требование четкости понятий, что предполагает избегание неопределенности и многозначности.
	
	\begin{enumerate}
		\item Философия языка: Аналитическая философия уделяет особое внимание анализу языка, как средства выражения и структурирования мыслей. Представители этого направления считают, что философские проблемы зачастую возникают из-за неправильного понимания или использования языка.
		
		\item Логический анализ: Аналитические философы рассматривают логический анализ как основной метод философского исследования. Важными аспектами являются формализация понятий и выведение логических выводов из четко определенных исходных утверждений.
		
		\item Философия науки: Аналитическая философия активно взаимодействует с наукой, особенно с логикой и математикой, а также уделяет внимание различию между научными и псевдонаучными утверждениями.
		
		\item Эмпиризм и критический рационализм: В рамках аналитической философии также проявляется традиция эмпиризма, где знания считаются результатом чувственного восприятия и логического анализа, а также рационализма, который подчеркивает важность разума и логики.
	\end{enumerate}
	
	Аналитическая философия развивается через несколько ключевых этапов, начиная с начала XX века и до наших дней. Этот процесс включает в себя как примирение с другими философскими традициями, так и критику тех подходов, которые аналитики считали излишне спекулятивными или неопределенными.
	
	В начале XX века аналитическая философия была ориентирована на логику и математику, с использованием логики для упорядочивания философских вопросов (например, работы Рассела и Фреге).
	
	В 1950-60-е годы аналитическая философия активно развивает концепции философии языка и философии действия, с акцентом на изучение того, как язык функционирует в повседневной жизни (работы Остина и позднего Витгенштейна).
	
	Современные аналитические философы часто работают в области философии сознания, философии науки, эпистемологии и этики, стремясь к более глубокому пониманию субъективных переживаний и методов научного познания. 
	
	Основные представители аналитической философии:
	\begin{enumerate}
		\item Бертран Рассел: Один из основателей аналитической философии, который занимался логикой и теорией познания. Рассел предложил концепцию логического атомизма, в которой мир рассматривался как совокупность атомарных фактов, а язык — как отражение этих фактов. Рассел утверждал, что философия должна заниматься анализом логических структур языка, чтобы выявить истинную природу вещей. Он также разработал теорию дескрипций, которая позволила точно определить значение предложений с указанием на объекты.
		
		\item Готлоб Фреге: Немецкий логик и философ, основатель современной логики и теории значений. Его работы заложили основы для развития аналитической философии, особенно в области логики и философии языка. Фреге ввел различие между смыслом (Sinn) и значением (Bedeutung) выражений, что стало важным вкладом в философию языка.
		
		\item Людвиг Витгенштейн: Важная фигура в развитии аналитической философии, чьи работы оказали большое влияние на философию языка и мышления. В своей ранней работе <<Логико-философский трактат>>  он предложил, что философия должна быть направлена на анализ языка и его логической структуры, а все философские проблемы являются следствием неправильного использования языка. В своей более поздней работе <<Философские исследования>> он развил концепцию <<языковых игр>>, утверждая, что значение слова определяется его использованием в конкретных контекстах.
	\end{enumerate}
	
	\par Несмотря на свой успех, аналитическая философия подвергалась критике со стороны представителей других философских течений. Одной из таких критик является утверждение, что излишняя концентрация на языке и логике ведет к игнорированию метафизических и экзистенциальных вопросов, которые являются важными для человеческого существования. Так, представители континентальной философии, такие как Жан-Поль Сартр и Мартин Хайдеггер, обвиняли аналитиков в чрезмерной абстракции и недостаточности внимания к реальным жизненным проблемам.
	
	Кроме того, существует мнение, что философия обыденного языка, предложенная Остином и Витгенштейном, ограничивает философские исследования в рамках банального повседневного опыта и не позволяет выйти за пределы обыденности.
	
	\section{Постпозитивизм: основные представители и идеи~\checkmark}
	Постпозитивизм — философское направление, возникшее в первой половине XX века как реакция на ограниченность классического позитивизма. Позитивизм утверждал, что единственно верным методом познания является эмпирическое наблюдение и опыт, а все знания, которые нельзя подтвердить через опыт, следует отвергнуть как ненаучные. Постпозитивизм критикует эту строгость, предлагая более гибкую и многозначную картину науки.
	\subsection{Основные идеи и представители постпозитивизма}
	\begin{enumerate}
		\item Критика верификационизма и эмпиризма. Постпозитивисты, в первую очередь, критикуют концепцию верификации, которая является основой классического позитивизма. Согласно этой концепции, утверждения, не поддающиеся эмпирической проверке, не имеют смысла. Однако постпозитивисты, такие как Карл Поппер и Томас Кун, утверждают, что наука не сводится к строгому эмпиризму.
		
		\item Поппер и принцип фальсифицируемости. Карл Поппер является одним из ключевых представителей постпозитивизма. Он предложил концепцию фальсифицируемости как основного критерия научности. В отличие от верификации, Поппер утверждал, что научные теории должны быть потенциально опровергнуты опытом, а не подтверждены. Теории, которые не могут быть опровергнуты, не являются научными. Это положение ставит под сомнение строгий эмпиризм, предлагая вместо этого более динамичную модель научного познания, где теория всегда поддается критическому пересмотру и испытанию.
		
		Несмотря на значимость идеи фальсификации, критики Поппера утверждают, что на практике многие научные теории трудно поддаются фальсификации, потому что ученые могут всегда найти способы защитить свои теории, адаптируя их к новым данным. Например, в некоторых случаях научные теории настолько эластичны, что они могут "поглотить" любое опровержение, что ставит под сомнение сам принцип фальсификации.
		
		\item Томас Кун и парадигмальные сдвиги. Томас Кун предложил концепцию научных революций и парадигм. Кун утверждал, что научное знание развивается не по прямой линии, а через смену парадигм — основополагающих теоретических моделей, которые определяют взгляды на мир в определенную эпоху. Когда накопление аномалий (данных, которые не согласуются с существующей теорией) становится слишком большим, происходит парадигмальный сдвиг, и старую модель заменяет новая. Это приводит к тому, что наука не является исключительно линейным процессом прогресса, а включает в себя и моменты радикальных изменений.
		
		Кун подвергся критике за то, что его концепция парадигмальных сдвигов ведет к относизму в науке, где каждая парадигма может быть верной в своем контексте, что делает невозможным объективное сопоставление разных теорий. Некоторые философы утверждают, что такое понимание научного прогресса угрожает универсальности научного знания и ведет к его фрагментации.
		
		\item Лакатос и методология научных исследований. Имре Лакатос предложил концепцию научных исследовательских программ, которые представляют собой группы теорий, объединенных общими принципами. В отличие от Поппера, который видел в фальсификации основной движущей силой науки, Лакатос акцентирует внимание на том, что в научных исследованиях теории не отвергаются полностью после каждой фальсификации, а подвергаются усовершенствованию и корректировке.
		
		Методология Лакатоса также подвергается критике, особенно в отношении его концепции научных программ. Оппоненты отмечают, что невозможно четко отделить <<жесткое ядро>> теории от её <<защитного пояса>>, что затрудняет практическое применение этой теории в реальной науке.
		
		\item Гадамер и философская герменевтика. Еще одним важным элементом постпозитивистской философии является герменевтика — теория и метод интерпретации текстов, предложенная Хансом-Георгом Гадамером. Он утверждал, что знание всегда связано с интерпретацией и контекстом, а значит, в научном познании нельзя игнорировать исторические и культурные предпосылки.
	\end{enumerate}
	
	\section{Философия постмодернизма~\checkmark}
	Постмодернизм — это философское и культурное течение, сформировавшееся во второй половине XX века как радикальная критика основ модерна: разума, истины, субъекта, прогресса, истории и универсализма. В философии постмодернизм представляет собой не столько единую систему взглядов, сколько стиль мышления, ориентированный на отказ от тотальности, иерархий, нормативных универсалий и метанарративов. Его ключевая установка — деконструкция великих нарративов и переосмысление способов производства смысла в культуре, языке и социальной практике.
	
	Постмодернизм возникает как реакция на модерн, то есть на философию Нового времени, Просвещения и классического рационализма. Он отказывается от следующих базовых установок модерна:
	\begin{enumerate}
		\item вера в рациональность как универсальное средство познания
		\item идея прогресса как поступательного развития человечества
		\item концепция автономного субъекта (Декарт, Кант)
		\item убеждённость в объективной истине и возможности её познания
		\item стремление к систематизации знания, универсальному объяснению и нормативной этике
	\end{enumerate}
	Исторические катастрофы XX века (мировые войны, Холокост, колониализм, кризис демократии) подорвали оптимизм модерна. Возникло философское сомнение в том, что разум и наука действительно ведут к свободе и гуманизму. На этом фоне философия постмодернизма поднимает вопрос: а что, если сами основания модерна ложны или тоталитарны по сути?
	\subsection{Основные черты философии постмодернизма}
	\begin{enumerate}
		\item Антиметанарративизм (Жан-Франсуа Лиотар): отказ от <<больших рассказов>> — то есть глобальных теорий, объясняющих всё (марксизм, либерализм, христианство, наука и др.).
		\item Деконструкция (Жак Деррида): анализ философских и культурных текстов с целью вскрытия внутренних противоречий, показ невозможности окончательного смысла.
		\item Смерть субъекта (Мишель Фуко, Жиль Делёз): отказ от идеи единого, автономного <<Я>> в пользу множественных, исторически и дискурсивно сконструированных форм идентичности.
		\item Симулякры и симуляции (Жан Бодрийяр): утверждение, что в постсовременности реальность вытесняется знаками, а <<симулякры>> становятся более реальными, чем сама действительность.
	\end{enumerate}
	
	\subsection{Основные философы постмодернизма и их концепции}
	
	\begin{enumerate}
		\item Жан-Франсуа Лиотар.
		
		В книге <<Состояние постмодерна>> (1979) Лиотар определяет постмодерн как <<недоверие к метанарративам>>. Он считает, что модерн создавал универсальные проекты — научные, политические, моральные, — однако все они в конечном итоге превращались в репрессивные формы власти. Постмодерн разрушает эти универсальные схемы, отдавая предпочтение локальным знаниям, малым нарративам, мозаичной культуре.
		
		\item Мишель Фуко
		
		Фуко исследует отношения между знанием и властью. В его концепции знание — не нейтральная истина, а форма власти, производящая субъекты и нормы. Он анализирует институции (тюрьму, больницу, школу), показывая, как они формируют дисциплинарное общество. Центральное понятие — дискурс, то есть исторически обусловленная система высказываний, которая определяет, что считается <<нормальным>>, <<истинным>> или <<безопасным>>.
		
		\item Жак Деррида
		
		Создатель метода деконструкции — аналитического подхода, направленного на выявление внутренних противоречий текстов и систем мышления. Деррида показывает, что смысл всегда отсрочен (развивая понятие différance), никогда не завершён и зависит от контекста. Он разрушает традиционные оппозиции: разум/чувства, речь/письмо, центр/периферия.
		
		\item Жиль Делёз и Феликс Гваттари
		
		В работах <<Анти-Эдип>> и <<Тысяча плато>> они развивают философию десубъективации, машин желания, ризомы (сетевой структуры знания). Они выступают против иерархий, предлагая видеть мир как множественную, нелинейную систему. Общество — не структура, а поток, состоящий из анонимных связей и сил.
		
		\item Жан Бодрийяр
		
		Развивает концепцию симулякров — копий, не имеющих оригинала. В постмодернистском мире, по Бодрийяру, реальность вытесняется гиперреальностью: медиа, реклама и цифровые технологии формируют <<виртуальные миры>>, где исчезает граница между реальным и мнимым. Политика, культура, идентичность превращаются в спектакль.
	\end{enumerate}
	
	Постмодернизм радикально переосмысляет классическую философию. Его аргументация строится на признании:
	\begin{enumerate}
		\item 	невозможности универсальных истин в условиях плюралистического мира;
		\item подозрения по отношению к авторитету разума;
		\item осознания исторической и лингвистической обусловленности знания.
	\end{enumerate}
	
	Однако постмодернизм подвергается и жёсткой критике:
	
	\begin{enumerate}
		\item Рационалистическая критика (Ю. Хабермас): постмодернизм, по мнению Хабермаса, — это форма <<непросвещённого антипросвещения>>. Отказ от истины и рациональности ведёт к релятивизму и неспособности к критической позиции.
		
		\item Этическая критика: если нет универсальных ценностей, как защищать права человека, бороться с несправедливостью или насилием?
		
		\item 	Критика изнутри: даже в самих постмодернистских текстах обнаруживается стремление к новым формам догмы (например, догма деконструкции, догма симулякра и т.д.).
		
	\end{enumerate}
	
	\section{Бытие как философская категория~\checkmark}
	
	Бытие — одно из центральных и наиболее общих понятий философии, обозначающее существование в самом широком смысле слова. Оно охватывает все существующее: реальное и мыслимое, материальное и духовное, актуальное и потенциальное. Раздел философии, изучающий бытие, его фундаментальные принципы, структуры и закономерности, называется \textit{онтологией}.
	
	\subsection{Историческое развитие понятия:}
	\begin{itemize}
		\item Впервые вопрос о бытии был поставлен в античной философии. Парменид из Элейской школы утверждал, что <<бытие есть, а небытия нет>>, так как мыслить и говорить можно лишь о том, что существует. Он приписывал бытию такие характеристики, как вечность, неизменность, однородность, неподвижность и единство.
		\item Для Платона истинным, подлинным бытием обладал трансцендентный мир идей (эйдосов) — вечных и неизменных умопостигаемых образцов. Чувственный мир вещей является лишь их тенью, находится между бытием и небытием, являясь <<полу-бытием>>.
		\item Аристотель критиковал Платона за <<удвоение мира>>. Он утверждал, что бытие имманентно миру и является прежде всего конкретной, индивидуальной вещью (субстанцией) как единство пассивной материи и активной формы.
		\item В Средние века сложилась иерархическая картина бытия. Высшим и подлинным бытием считался Бог (\textit{ens perfectissimum}), который есть чистый акт бытия и является источником бытия всего сотворенного мира (творение из ничего — \textit{creatio ex nihilo}).
		\item В философии Нового времени бытие стало рассматриваться через призму познающего субъекта (гносеологический поворот). Бытие отождествлялось либо с природой, материей (\textit{материализм}), либо с сознанием, мышлением (\textit{идеализм}). Р. Декарт разделил бытие на две субстанции: мыслящую (\textit{res cogitans}) и протяженную (\textit{res extensa}).
		\item Гегель в своей диалектике показал, что чистое, абсолютно неопределенное бытие тождественно чистому ничто. Их единство и взаимопереход порождают первую конкретную категорию — становление.
		\item В философии XX века Мартин Хайдеггер предпринял попытку вернуться к <<вопросу о смысле бытия>>. Он упрекал всю западную философию в <<забвении бытия>>, в том, что она подменяла вопрос о бытии вопросом о сущем. Он проводил онтологическое различие между <<бытием>> (\textit{Sein}) как таковым и <<сущим>> (\textit{Seiendes}) — конкретными вещами. По Хайдеггеру, человек (\textit{Dasein} — <<вот-бытие>>) является тем единственным сущим, которое способно задаваться вопросом о бытии.
		\item Жан-Поль Сартр, другой экзистенциалист, противопоставлял \textit{бытие-в-себе} (неподвижное, косное бытие вещей) и \textit{бытие-для-себя} (сознание человека), которое характеризуется свободой и способностью к самоопределению (<<существование предшествует сущности>>). Для человека это означает, что он изначально не является ничем и должен сам себя сотворить через свои выборы, неся за них абсолютную ответственность.
	\end{itemize}
	
	
	\subsection{Основные формы бытия:}
	В современной философии принято выделять следующие основные формы бытия:
	\begin{itemize}
		\item Бытие вещей, процессов, состояний природы (объективная реальность, первая природа).
		\item Бытие человека (экзистенция; существование человека как единства телесного и духовного).
		\item Бытие духовного (идеального), которое делится на:
		\begin{itemize}
			\item индивидуализированное (субъективная реальность: сознание, самосознание, бессознательное);
			\item объективированное (интерсубъективная реальность: социальные нормы, ценности, научные знания, искусство).
		\end{itemize}
		\item Бытие социального, которое делится на бытие отдельного человека в обществе и бытие самого общества (вторая, искусственная природа).
		\item В последнее время выделяют также бытие виртуального как особую форму реальности, порожденную компьютерными технологиями и обладающую свойствами порожденности, актуальности и автономности.
	\end{itemize}
	
	\section{Философское понимание материи~\checkmark}
	
	Материя (от лат. \textit{materia} — вещество) — это философская категория для обозначения объективной реальности, которая является субстратом всех существующих в мире вещей и явлений, их свойств и связей. В марксистской философии (диалектическом материализме) В.И. Ленин дал ставшее классическим гносеологическое определение: материя — это <<объективная реальность, которая дана человеку в ощущениях его, которая копируется, фотографируется, отображается нашими ощущениями, существуя независимо от них>>.
	
	\subsection{Эволюция представлений о материи:}
	\begin{itemize}
		\item В античности материя чаще всего понималась как вещество, праоснова, из которой состоят все вещи (вода у Фалеса, апейрон у Анаксимандра, атомы у Демокрита). Для Платона материя — это почти небытие, пассивное инертное начало (<<кормилица>>), которое принимает формы-эйдосы. Аристотель видел в материи одну из четырех причин, чистую пассивную возможность (\textit{dynamis}), которая становится действительной вещью (\textit{energeia}) только в соединении с активной формой.
		\item В философии Нового времени (XVII-XVIII вв.) в рамках механистического материализма материя отождествлялась с веществом (\textit{substance}) и сводилась к его неизменным, атрибутивным свойствам: протяженности, весу, непроницаемости, инертности (И. Ньютон).
		\item Кризис в физике на рубеже XIX-XX вв. (открытие электрона, радиоактивности, релятивистских эффектов) привел к краху механистической картины мира и так называемому <<исчезновению материи>> для старого подхода. Философский идеализм (физический идеализм) использовал это как повод для утверждения, что <<материя исчезла>>, а на самом деле существуют лишь <<комплексы ощущений>>. Диалектический материализм ответил на это введением более широкого, философского определения материи как объективной реальности, подчеркивая, что исчез не сам материальный мир, а лишь старый предел его познания.
	\end{itemize}
	
	\subsection{Современное понимание:}
	С точки зрения современной науки и философии, материя обладает сложной системной организацией и неисчерпаемостью своих свойств.
	\begin{itemize}
		\item Виды материи: вещество, физическое поле (электромагнитное, гравитационное), физический вакуум. Открытие поля как вида материи показало, что материя может быть не только дискретной, но и непрерывной (континуальной).
		\item Уровни организации: неживая природа (элементарные частицы, атомы, молекулы, макротела, планеты, галактики), живая природа (клетки, организмы, популяции, биосфера), социально-организованная материя (человек, общество).
		\item Атрибуты материи: важнейшими неотъемлемыми свойствами материи, способами ее существования являются \textit{движение, пространство и время}.
	\end{itemize}
	
	
	\section{Движение, развитие и диалектика~\checkmark}
	
	Движение в самом широком философском смысле — это способ существования материи, любое изменение вообще. Движение включает в себя все происходящие во вселенной процессы: от простого перемещения до мышления. Движение абсолютно, в то время как покой относителен и представляет собой момент движения (например, равновесие системы или сохранение качественного состояния).
	
	\subsection{Формы движения материи:}
	Фридрих Энгельс предложил классификацию форм движения материи, основанную на принципе развития от низшего к высшему. Эта классификация имеет важное методологическое значение против редукционизма (сведения высших форм к низшим).
	\begin{enumerate}
		\item Механическая (простое перемещение тел в пространстве).
		\item Физическая (теплота, электричество, гравитация, внутриатомные процессы).
		\item Химическая (соединение и разъединение молекул, химические реакции).
		\item Биологическая (обмен веществ, наследственность, эволюция видов).
		\item Социальная (история человеческого общества, деятельность людей).
	\end{enumerate}
	Каждая высшая форма движения включает в себя низшие в <<снятом>> виде, но не сводится к ним.
	
	\subsection{Развитие и его концепции:}
	Развитие — это частный случай движения, представляющий собой необратимое, направленное, закономерное изменение материальных и идеальных объектов. Результатом развития является возникновение нового качественного состояния объекта (прогресс или регресс).
	Существуют две основные концепции (модели) развития:
	\begin{itemize}
		\item Метафизическая: Рассматривает развитие как простое количественное изменение (увеличение или уменьшение), повторение. Источник движения видится во внешнем толчке. Движение понимается как движение по замкнутому кругу.
		\item Диалектическая: Понимает развитие как саморазвитие, источником которого являются внутренние противоречия объекта. Развитие идет по спирали, сочетая в себе поступательность, преемственность и кажущийся возврат к старому на новом, более высоком уровне.
	\end{itemize}
	
	\subsection{Законы диалектики (по Гегелю):}
	Ключевые принципы диалектического понимания развития — это всеобщие законы диалектики:
	\begin{itemize}
		\item \textbf{Закон единства и борьбы противоположностей} (раскрывает источник развития — внутренние противоречия).
		\item \textbf{Закон перехода количественных изменений в качественные} (раскрывает механизм развития — постепенное накопление изменений приводит к скачку, смене качества).
		\item \textbf{Закон отрицания отрицания} (раскрывает направление и форму развития — спиралевидность, преемственность). Центральным понятием здесь является \textit{снятие} (нем. Aufhebung) — отрицание с удержанием всего положительного, что было в старом качестве.
	\end{itemize}
	
	\subsection{Современные концепции развития:}
	В XX веке возникла \textit{синергетика} (Г. Хакен, И. Пригожин), которая изучает процессы самоорганизации в сложных, открытых, нелинейных системах. Она дополнила диалектику понятиями \textit{бифуркации} (точки выбора пути развития, где случайность играет ключевую роль) и \textit{аттрактора} (относительно устойчивого состояния, к которому стремится система).
	
	\section{Пространство и время~\checkmark}
	
	Пространство и время — это всеобщие формы существования материи. Они являются фундаментальными категориями, описывающими структурность (протяженность, порядок расположения элементов) и длительность (порядок смены состояний) бытия.
	
	\subsection{Основные концепции пространства и времени:}
	\begin{enumerate}
		\item Субстанциальная концепция: Пространство и время рассматриваются как самостоятельные сущности (субстанции), существующие независимо от материи и ее движения. Пространство — это пустое бесконечное вместилище, абсолютно неподвижное и однородное. Время — это чистая длительность, которая течет одинаково во всей Вселенной. Эта концепция была характерна для Демокрита и в классической форме разработана Исааком Ньютоном.
		\item Реляционная концепция (от лат. \textit{relatio} — отношение): Пространство и время не существуют сами по себе, а являются системами отношений между материальными объектами и процессами. Пространство — это выражение порядка сосуществования объектов, а время — порядка смены их состояний. Нет материи — нет ни пространства, ни времени. Основоположником этой концепции считается Аристотель, а в Новое время — Готфрид Лейбниц.
	\end{enumerate}
	
	\subsection{Философские интерпретации:}
	В философии Иммануила Канта пространство и время не являются ни субстанциями, ни отношениями вещей самих по себе (\textit{ноуменов}). Это априорные (доопытные) формы нашей чувственности. Мы не можем воспринимать мир иначе, как в пространстве и времени, потому что так устроен наш познавательный аппарат, который структурирует поток ощущений.
	
	\subsection{Современные научные представления:}
	Общая и специальная теории относительности А. Эйнштейна экспериментально и теоретически подтвердили и углубили реляционную концепцию.
	\begin{itemize}
		\item Они показали, что свойства пространства и времени зависят от скорости движения и распределения масс и энергии.
		\item Пространство и время образуют неразрывное единство — четырехмерный \textit{пространственно-временной континуум}, который <<искривляется>> вблизи массивных тел.
		\item Свойства времени (например, его темп) относительны и зависят от скорости движения системы отсчета и гравитационных полей (гравитационное и релятивистское замедление времени).
		\item Актуальной остается проблема <<стрелы времени>> — объяснения его необратимости, которая чаще всего связывается со вторым законом термодинамики (законом возрастания энтропии).
	\end{itemize}
	
	\section{Причинность. Детерминизм и свобода~\checkmark}
	
	\textit{Причинность} (каузальность) — это философская категория, отражающая объективную генетическую связь между явлениями, при которой одно явление (\textbf{причина}) при определенных условиях с необходимостью порождает, вызывает к жизни другое явление (\textbf{следствие}). Ключевые черты причинности: объективность, всеобщность, необходимость и временная асимметрия (причина предшествует следствию).
	
	\subsection{Детерминизм:}
	Детерминизм (от лат. \textit{determinare} — определять) — это философское учение о всеобщей закономерной связи и причинной обусловленности всех явлений.
	\begin{itemize}
		\item Механистический (лапласовский) детерминизм: крайняя форма, утверждавшая, что зная точное положение и скорость всех частиц во вселенной в данный момент, можно однозначно предсказать все ее будущее и восстановить все прошлое. Эта модель отрицает объективный характер случайности.
		\item Телеологический детерминизм: учение о предопределенности событий некой целью (\textit{telos}), божественным промыслом или <<мировым духом>> (Платон, Лейбниц, Гегель).
	\end{itemize}
	
	\subsection{Критика и современные взгляды:}
	\begin{itemize}
		\item Дэвид Юм подверг концепцию причинности скептической критике. Он считал, что мы наблюдаем не саму причинную связь, а лишь привычную регулярную последовательность событий (<<после этого>>, но не <<вследствие этого>>). Следовательно, причинность — это лишь привычка нашего ума (ассоциация идей), а не объективное свойство мира.
		\item Иммануил Кант, отвечая Юму, утверждал, что причинность — это не привычка, а априорная категория нашего рассудка, которая позволяет упорядочивать опыт и превращать его в научное знание.
		\item \textit{Современный диалектический детерминизм} преодолел крайности механистического. Он признает существование разных типов связей и объективный характер не только необходимости, но и случайности (которая является формой проявления необходимости или результатом пересечения независимых причинных рядов).
		\item Признается существование не только жестких, динамических закономерностей, но и статистических (вероятностных) закономерностей (например, в квантовой механике, термодинамике, социальных процессах). Статистический закон определяет не однозначное поведение отдельного элемента, а распределение вероятностей и устойчивое поведение всей их совокупности.
	\end{itemize}
	
	\subsection{Проблема свободы и детерминизма:}
	\textit{Индетерминизм} — это концепция, отрицающая всеобщий характер причинности.
	\begin{itemize}
		\item Волюнтаризм: крайняя форма индетерминизма в этике, утверждающая абсолютную свободу воли, не обусловленную никакими причинами.
		\item Фатализм: крайняя форма детерминизма, утверждающая полную предопределенность всех поступков человека и отрицающая свободу выбора. В отличие от каузального детерминизма, фатализм может утверждать, что некое событие произойдет независимо от предшествующих ему событий.
	\end{itemize}
	Решение этой дилеммы часто ищут в \textit{совместимости (компатибилизме)}: свобода понимается не как отсутствие причин, а как способность действовать на основе познанной необходимости, в соответствии со своими собственными целями и ценностями. <<Свобода есть познанная необходимость>> (Спиноза, Гегель, Энгельс). В экзистенциализме свобода понимается как бремя выбора, от которого человек не может уклониться.
	
	\section{Проблема сознания в философии~\checkmark}
	Проблема сознания в философии — одна из наиболее глубоких и многогранных тем, которая затрагивает вопросы о сущности человеческого опыта, природе восприятия, роли субъекта и объекта, а также о границах познания. Эту проблему исследуют различные философские школы и направления, начиная от древнегреческих философов и до современных представителей нейронауки и философии разума. В рамках философии сознания рассматриваются такие вопросы, как: что есть сознание, как оно связано с телом, какие его свойства и как его можно объяснить с точки зрения философии и науки?
	
	\subsection{Историческое развитие проблемы сознания}
	Древнегреческие философы, такие как Платон и Аристотель, в значительной степени определяли подходы к пониманию природы человека и его разума. Платон рассматривал сознание как нечто небесное, принадлежащее миру идей, отделяя материю от разума. Аристотель, в свою очередь, стремился найти более земное объяснение, связывая разум с деятельностью души, которая, по его мнению, определяет все формы жизни.
	
	В Средневековье философы, такие как Августин Блаженный и Фома Аквинский, сосредоточились на связи сознания с Божественным началом. Августин считал, что "Cogito, ergo sum" (Я мыслю, следовательно, я существую) выражает фундаментальную сущность человека — способность мыслить, познавать себя и мир. Он подчеркнул важность самосознания в познании реальности и Бога.
	
	С развитием научной мысли в Новое время внимание к проблеме сознания усиливается. Рене Декарт, например, выделял сознание как определяющую особенность человеческой сущности, противопоставляя его материальной реальности (двоичное деление на res cogitans и res extensa). Суть его подхода в том, что мышление и сознание являются основанием для всего познания, а материальный мир может быть постигнут через научный метод.
	
	В 19 и 20 веках философы, такие как Иммануил Кант и Фридрих Ницше, размышляли о природе сознания, но их подходы уже разительно отличались от прежних. Кант утверждал, что наше познание ограничено тем, как мы воспринимаем мир, что можно назвать априорными формами сознания, а Ницше сосредотачивался на более критическом осмыслении роли разума в жизни человека, часто подчеркивая иррациональные и бессознательные аспекты.
	
	\subsection{Основные философские подходы к проблеме сознания}
	\begin{enumerate}
		\item Дуализм (Рене Декарт) — одно из самых известных решений проблемы сознания. Дуализм утверждает, что существует два принципиально разных начала: физический мир и мир разума (сознания). Декарт, например, утверждал, что физическое тело и разум или сознание принадлежат к разным сущностям. Тело подчиняется законам механики, в то время как сознание является нематериальным и независимым от физической реальности.
		
		
		Проблема взаимодействия между материей и нематериальной душой или сознанием, что не имеет научного объяснения.
		Современные философы, такие как Даниэль Деннет, указывают, что дуализм приводит к парадоксу: как могут взаимодействовать две совершенно разные субстанции?
		
		
		\item Материализм — противоположный дуализму подход, утверждающий, что сознание не может существовать без материи. Этот подход считает, что все психические явления можно объяснить с точки зрения нейрофизиологии и функционирования мозга. Представители материализма, такие как Джон Серль и Даниэль Деннет, подчеркивают, что сознание является продуктом активности мозга, и любые ощущения, переживания или мысли могут быть объяснены через биологические процессы.
		
		
		Множество философов, в том числе Томас Нагель и Дэвид Чалмерс, утверждают, что субъективный опыт (qualia) невозможно полностью свести к биологическим процессам. Проблема сознания в том, что как бы мы ни исследовали мозг, мы не можем точно объяснить, как возникает сам субъективный опыт восприятия.
		
		\item Функционализм (Джереми Бентам, Даниэль Деннет) — подход, при котором сознание рассматривается не как нечто материальное, но как совокупность функциональных процессов. С точки зрения функционализма, важным является не то, как устроен мозг, а каковы его функции и процессы, которые создают сознание. В этом смысле сознание можно рассматривать как компьютерную программу, которая исполняет определенные алгоритмы.
		
		
		Критики функционализма (например, философы, такие как Томас Нагель) утверждают, что важно не только что делает система, но и как она это делает, то есть какие субъективные переживания сопровождают эти процессы. Функционализм не объясняет, почему какой-либо процесс является сознательным.
		
		\item Панпсихизм — теория, утверждающая, что сознание является свойством всех сущностей во Вселенной, даже тех, которые традиционно не считаются живыми (например, камни или атомы). Это философское направление предлагает радикальный взгляд на проблему сознания, считая его фундаментальным аспектом реальности, а не результатом сложной организации материи.
		
		
		\par Панпсихизм сталкивается с проблемой объяснения, каким образом низкие уровни сознания (например, у атомов) могут объединяться в более сложные формы сознания, что вызывает вопросы о сложности и целесообразности этой теории.
	\end{enumerate}
	\par С развитием нейронауки и когнитивной науки проблема сознания продолжает привлекать внимание философов. Вопросы о том, как нейрофизиология и сознание соотносятся, актуальны для таких теорий, как нейробиологизм и психофизический редукционизм. Современные ученые и философы продолжают исследовать, как мозг генерирует сознание и как оно связано с индивидуальными переживаниями, в то время как такие подходы, как нейрофеноменология, пытаются интегрировать субъективный опыт с объективными данными.
	
	\section{Субъект и объект познания. Познавательное отношение человека к миру}
	
	\section{Чувственное и рациональное познание~\checkmark}
	Проблема чувственного и рационального познания занимает центральное место в истории философии. Она касается вопросов о том, как мы познаём мир, какие источники познания для нас доступны и каким образом их можно соотнести. Вопрос о соотношении чувственного и рационального познания представляет собой одну из ключевых философских проблем, обсуждавшихся с античных времён и до современности.
	
	Чувственное познание представляет собой восприятие мира через органы чувств — зрение, слух, обоняние, осязание и вкус. Это первичный источник знания о мире, который даёт нам непосредственные данные о явлениях природы и объектах окружающей действительности. Чувственное познание всегда связано с конкретностью, эмпирической доказательностью и ограничено рамками тех условий, в которых оно происходит. Оно носит, в силу своей эмпирической природы, скорее чувственный, чувственно-наглядный характер, и, несмотря на свою первичную важность, может быть ошибочным или неполным.
	
	Рациональное познание связано с разумом и мышлением, с логическим осмыслением данных, полученных через чувственное восприятие. Рациональное познание включает в себя рассуждения, обобщения, выдвижение гипотез, построение теорий, и предполагает использование понятий, абстракций, и систематизации. Рациональность позволяет не только анализировать данные чувственного опыта, но и строить выводы, которые не всегда очевидны на основе только чувственного восприятия.
	
	В истории философии соотношение чувственного и рационального познания приобретало разные формы. Одной из первых попыток систематизации этих двух видов познания можно найти у древнегреческих философов.
	
	\begin{enumerate}
		\item Платон утверждал, что чувственное познание всегда связано с иллюзиями и заблуждениями. Согласно ему, мир, который мы воспринимаем, является лишь тенью истинного мира идей, который доступен лишь через разум. Чувственное восприятие для Платона служит лишь начальным шагом, но не является источником истинного знания.
		
		\item Аристотель, напротив, утверждал, что знание начинается с чувственного восприятия, которое затем подвергается обработке разумом. Он разработал концепцию эмпиризма, согласно которой мир можно познавать через наблюдение и опыт. Для Аристотеля восприятие мира через чувства имеет первостепенное значение, но в то же время оно нуждается в анализе и систематизации через разум.
		
		\item Декарт в своём подходе к познанию сделал акцент на сомнении как методе, который позволял ему приходить к уверенности в существовании себя как мыслящего субъекта: <<Cogito, ergo sum>> (мыслю, следовательно, существую). Он ставил разум на первое место, утверждая, что чувственное восприятие может быть обманчивым, и только разум способен достигать истинного знания.
		
		\item Иммануил Кант сделал попытку примирить эти два подхода. Он утверждал, что наши знания о мире всегда являются результатом взаимодействия чувственного восприятия и категорий разума. Кант разделял мир на <<вещи сами по себе>>, которые мы не можем познать напрямую, и <<вещи, как они нам представляются>>, которые являются продуктом нашей познавательной деятельности. По Канту, чувственное восприятие без разума было бы слепым, а разум без чувственного опыта — пустым.
		
		\item Гегель пошёл ещё дальше, развивая философскую систему, в которой познание является процессом, в котором разум развивается через чувственное восприятие, а затем возвышается к более высокому, диалектическому пониманию. Для Гегеля процесс познания — это не статический момент, а движение, где чувственное восприятие служит начальной стадией в развитии идеи.
	\end{enumerate}
	
	Современная философия продолжает обсуждать соотношение чувственного и рационального познания. Проблема перешла в контексты науки, психологии и когнитивных наук. Современные философы и учёные обсуждают, как восприятие и сознание соотносятся с объективной реальностью, и как разум может корректировать или искажать чувственное восприятие.
	
	В области эмпиризма важно отметить работы таких философов, как Давид Юм, который подчёркивал, что весь опыт и знания происходят из чувственного восприятия, а всё, что выходит за рамки непосредственного опыта, является продуктом разума. Его критика рационализма заключается в том, что человеческие познания не могут быть абсолютными, а все наши обобщения базируются на привычках и опыте, а не на универсальных истинах.
	
	С другой стороны, представители рационалистической традиции, такие как Гегель или более современные философы, утверждают, что разум, даже если и основывается на данных чувственного опыта, выходит за рамки того, что мы воспринимаем органами чувств. К примеру, в гегелевской диалектике познание развивается через противоречия и синтез, создавая новые формы знания.
	
	Соотношение чувственного и рационального познания можно рассматривать как один из важнейших вопросов философской эпистемологии. Вопрос заключается в следующем: может ли разум быть независим от чувственного восприятия? С одной стороны, научные открытия и рациональные построения требуют эмпирического подтверждения, с другой — рациональные структуры, такие как математика или логика, развиваются независимо от физического опыта.
	
	Кроме того, существует проблема субъективности чувственного восприятия. Каждый человек воспринимает мир по-своему, и вопросы о том, как избежать ошибок в интерпретации чувственного опыта, становятся актуальными для философии науки и эпистемологии. Какой степени достоверности могут достигать выводы, основанные на чувственном опыте? И как обеспечить объективность, если восприятие мира всегда субъективно?
	
	\section{Основные концепции истины}
	
	\section{Многообразие философских концепций человека}
	
	\section{Научное познание и его специфические признаки. Критерии научности~\checkmark}
	
	Научное познание — это процесс получения и систематизации знаний о мире с помощью объективных методов, основанных на фактических данных, логике и опыте. В отличие от других форм познания, таких как мифологическое, религиозное или обыденное, научное познание ориентировано на выявление объективных законов природы и общества, проверяемых экспериментально и логически. Оно базируется на конкретных методах исследования и систематическом подходе, что делает его более строгим и формализованным.
	
	\subsection{Специфические признаки научного познания}
	\begin{enumerate}
		\item Объективность
		Научное познание направлено на получение знаний, которые не зависят от субъективных мнений и личных пристрастий исследователя. Оно должно отражать объективную реальность, а не индивидуальные восприятия. Это означает, что результаты научных исследований могут быть воспроизведены другими учеными при соблюдении тех же условий.
		
		\item Системность
		Научное познание включает в себя не просто отдельные факты, но и их систематизацию в целостные теории и законы. Это позволяет создавать логически стройную картину мира, которая объясняет различные явления, их взаимосвязь и закономерности.
		
		\item Эмпиричность
		Научные знания основываются на фактах, полученных в ходе наблюдений, экспериментов и измерений. Эмпирический характер науки заключается в том, что теории и гипотезы подлежат проверке на практике. Без эмпирического подтверждения теория считается неполной.
		
		\item Проверяемость и воспроизводимость
		Научные теории и эксперименты должны быть проверяемыми. Это означает, что они должны быть такими, чтобы их результаты можно было подтвердить или опровергнуть через эксперименты или дальнейшие исследования. Важным аспектом является возможность воспроизведения экспериментов другими учеными, что гарантирует объективность и независимость полученных результатов.
		
		\item Рациональность и логичность
		Научное познание использует разум и логику как основные инструменты для анализа, синтеза и построения теорий. Оно основано на логических выводах, которые обеспечивают стройность и последовательность в объяснении явлений. Каждый этап научного познания должен быть логически обоснован и непротиворечив.
		
		\item Теоретичность
		Научное познание стремится не только к сбору фактов, но и к выведению теорий, которые объясняют эти факты. Теория в науке служит связующим звеном между эмпирическими данными и закономерностями, объясняя причины явлений и предсказывая их развитие в будущем.
		
		\item Прогностичность
		Наука не только объясняет и описывает явления, но и делает возможным предсказание их будущих изменений. Это достигается благодаря выявлению закономерностей, которые позволяют строить прогнозы о развитии событий или явлений в будущем.
		
		\item Общезначимость
		Научные утверждения имеют универсальный характер. Они не привязаны к конкретным условиям, индивидуальным особенностям или временным рамкам. Если научное утверждение истинно, то оно истинно в любых условиях и для всех субъектов исследования.
	\end{enumerate}
	
	\subsection{Критерии научности}
	Критерии научности определяют, что именно является научным, а что — ненаучным или псевдонаучным. Эти критерии помогают выделить научные знания среди других форм познания.
	\begin{enumerate}
		\item Критерий эмпирической проверки
		Научное утверждение должно быть эмпирически проверяемым. Это означает, что гипотезу или теорию можно подвергнуть проверке с помощью экспериментов или наблюдений, которые могут подтвердить или опровергнуть ее.
		
		\item Критерий логической обоснованности
		Научные утверждения должны быть логически стройными и последовательными. Это предполагает, что выводы, сделанные на основе теории, должны логически вытекать из исходных положений и не содержать внутренних противоречий.
		
		\item Критерий воспроизводимости
		Научные исследования должны быть воспроизводимыми, то есть другие исследователи, проводя аналогичные эксперименты при тех же условиях, должны получать аналогичные результаты.
		
		\item Критерий предсказательной силы
		Научная теория должна обладать способностью предсказывать новые факты или явления. Чем точнее и более универсальные предсказания делает теория, тем более научной она считается.
		
		\item Критерий интерсубъективности
		Научные утверждения должны быть интерсубъективными, то есть согласованными с точки зрения разных исследователей. Результаты научного исследования должны быть такими, чтобы их можно было воспринимать и интерпретировать различными учеными в разных культурах и странах.
		
		\item Критерий фальсифицируемости (по Попперу)
		Согласно философу Карлу Попперу, научная теория должна быть фальсифицируемой, то есть должна содержать такие предсказания, которые могут быть опровергнуты экспериментом или наблюдением. Ненаучные теории не допускают опровержения, а значит, не могут быть признаны научными.
	\end{enumerate}
	
	Научное познание отличается от других форм познания, таких как религиозное, мифологическое, художественное и обыденное. Религиозное познание часто основывается на вере и догматах, мифологическое познание имеет символический и метафорический характер, а обыденное познание формируется через личный опыт и интуицию, часто не систематизируясь и не подтверждаясь логическими аргументами или экспериментами.
	
	Научное познание отличается от этих форм строгими критериями, такими как объективность, эмпирическая проверяемость, логическая строгость и возможность воспроизведения результатов. Оно не ограничивается субъективным восприятием или традициями, а стремится к универсальности и объективности.
	
	\section{Общество как объект философского познания~\checkmark}
	Общество понимается как продукт целенаправленной и разумно организованной совместной деятельности больших групп людей, объ­единенных не на основе общности, а на основе совместных интере­сов и договоре. Общество — объективная реальность, форма существования бы­тия, обладающая внутренней структурой, целостностью, законами, на­правленностью развития.
	
	Сложный характер развития общества определяется его структу­рой. Экономическая сфера включает в себя производство, распреде­ление, обмен и потребление материальных благ. Она выступает как экономическое пространство, в котором организуется хозяйственная жизнь страны, осуществляется взаимодействие всех отраслей эконо­мики, а также международное экономическое сотрудничество.
	
	Социальная сфера — это сфера общественной жизни, в рамках которой удовлетворяются прямые жизненные потребности членов об­щества, а также происходит взаимодействие различных общностей людей. Политическая сфера жизни общества — сфера отношений между на­циями и другими большими социальными группами по поводу государ­ственной власти и государственного устройства внутри данного общест­ва, а также отношений между государствами на международной арене.
	\subsection{Общество в истории философии}
	Философское осмысление общества началось ещё в древнегреческой философии. Уже Платон и Аристотель поднимали вопросы о справедливом государстве, природе власти и роли гражданина. Платон в <<Государстве>> рассматривал общество как отражение порядка в душе: идеальное государство соответствует гармонии между разумом, волей и желаниями. Аристотель, в свою очередь, определял человека как <<политическое животное>>, подчёркивая, что вне общества человек не может реализовать свою природу. Общество у Аристотеля — это естественный союз, возникающий из потребностей, но имеющий целью благо.
	
	В средневековой философии (Августин, Фома Аквинский) общество осмысляется сквозь призму христианской теологии. Основное различие проводится между <<городом земным>> (обществом грешников, строящим мир на власти и насилии) и <<городом Божьим>> (сообществом праведников). Таким образом, философия Средневековья утверждает трансцендентный источник общественного порядка, противопоставляя его человеческой самонадеянности.
	
	С приходом Нового времени философия общества приобретает рационалистический характер. У Томаса Гоббса в <<Левиафане>> общество возникает как искусственный механизм, созданный для защиты от насилия и смерти. Естественное состояние — это война всех против всех, а общественный договор становится способом создания государства, обеспечивающего безопасность. Джон Локк, напротив, утверждает, что общество строится на естественных правах человека — на свободе, собственности и равенстве — и что государство должно быть ограничено этими правами. В обоих случаях общество — продукт договора, но с разными основаниями: для Гоббса — страх, для Локка — свобода.
	
	Руссо радикализирует теорию общественного договора, утверждая, что общество становится несправедливым, как только возникает частная собственность, и только возвращение к <<общей воле>> может обеспечить подлинную свободу. Таким образом, философия начинает видеть в обществе не только порядок, но и источник отчуждения, подавления, насилия.
	
	\subsection{Классическая немецкая философия: идеализм и диалектика общества}
	Особое значение имеет немецкая классическая философия. У Гегеля общество предстает как реализация разума в истории. В своей <<Философии права>> он вводит различие между гражданским обществом (сферой частных интересов, конкуренции, семьи и труда) и государством (выражением всеобщего разума). Гегель не противопоставляет общество и государство, а видит в их развитии логику становления свободы.
	
	Однако Карл Маркс, критикуя Гегеля, утверждает, что философия должна исходить не из абстрактного разума, а из реального материального основания — производства. В его концепции общество есть совокупность производственных отношений, определяемых уровнем развития производительных сил. Базис — это экономика, а надстройка — это право, мораль, идеология. Маркс предлагает диалектический и материалистический метод анализа общества, в котором его развитие есть результат внутренних противоречий: классовая борьба между угнетёнными и угнетателями. Его теория отчуждения показывает, как в условиях капитализма труд становится внешним и враждебным человеку.
	
	\subsection{Современная философия общества: от структурализма к постмодернизму}
	В XX веке философское познание общества переживает новый этап. Появляются такие направления, как феноменология, экзистенциализм, структурализм и постмодернизм. Каждый из них по-своему осмысляет общество.
	
	У Эммануэля Левинаса и Жана-Поля Сартра общество — это не просто совокупность индивидов, а поле этических отношений. Для Левинаса общественное начинается с <<лица Другого>> — с этического требования, которое невозможно проигнорировать. Для Сартра, напротив, общество — это результат экзистенциального выбора и конфликта между свободами: <<ад — это другие>>.
	
	Мишель Фуко рассматривает общество как систему власти и дисциплинарных практик, формирующих субъекта. В <<Надзирать и наказывать>> он показывает, как институты (школа, тюрьма, больница) производят не только знания, но и подчинение. Таким образом, общество — это сеть дискурсов и норм, регулирующих поведение людей. Подобный подход продолжает постструктуралистская традиция, утверждающая, что <<человека как такового>> не существует — он является результатом социальных и языковых конструкций.
	
	Философия общества ставит ряд ключевых вопросов:
	\begin{enumerate}
		\item Что первично — индивид или общество? (индивидуализм против холизма)
		\item Является ли общество естественным или искусственным образованием?
		\item Каковы механизмы воспроизводства власти и социальной иерархии?
		\item Возможно ли справедливое общество, и на каких основаниях оно может быть построено?
	\end{enumerate}
	
	\par Каждая философская школа отвечает на эти вопросы по-своему. Например, либерализм (Локк, Милль, Ролз) настаивает на приоритете личности и правах человека. Коммунитаризм (Макаинтайр, Тейлор) указывает, что личность формируется только в рамках общины и традиций. Марксизм видит общество как арену классовой борьбы и отчуждения. Феминистская философия критикует общественные институты как структурно патриархальные. Этика дискурса (Хабермас) предлагает диалог как основу социального консенсуса.
	
	\section{Философское изучение культуры~\checkmark}
	\subsection{Основные подходы}
	\begin{itemize}
		\item Аксиологический (ценностный) подход: Рассматривает культуру как иерархию ценностей, которые определяют цели и нормы человеческой деятельности. Представители (Г. Риккерт, М. Вебер) считали, что именно ценности (истина, добро, красота и др.) придают смысл и направленность человеческому существованию и истории. Культура --- это совокупность всего, что создано человеком ради осуществления и благодаря общезначимым ценностям.
		\item Деятельностный подход: Определяет культуру как совокупность способов и результатов человеческой деятельности, как технологию воспроизводства и изменения социальной жизни. Культура в этом понимании --- это не только совокупность продуктов труда (артефактов), но и сама "технология" деятельности, включающая в себя знания, навыки, нормы и образцы поведения, которые передаются из поколения в поколение и позволяют человеку преобразовывать мир и самого себя.
		\item Семиотический подход (культура как система знаков): Интерпретирует культуру как знаковую систему, подобную языку, через которую передаются социальные смыслы и значения. Культура понимается как совокупность "текстов" (в широком смысле --- от ритуала до произведения искусства), которые несут в себе информацию. Юрий Лотман ввел понятие \textbf{семиосферы} --- единого семиотического пространства культуры, внутри которого возможны коммуникация и порождение новых смыслов. Семиосфера неоднородна и состоит из "ядра" и "периферии", а ее развитие происходит за счет диалога между различными культурными "языками".
	\end{itemize}
	
	\subsection{Ключевые фигуры и концепции}
	\begin{itemize}
		\item Н. Я. Данилевский ("Россия и Европа"): Автор теории \textbf{"культурно-исторических типов"} (цивилизаций). Он считал, что человечество делится на локальные цивилизации (египетскую, китайскую, греческую, романо-германскую, славянскую и др.), которые, подобно живым организмам, проходят стадии рождения, роста, расцвета, упадка и гибели. Основы одной цивилизации непередаваемы другой. Взаимодействие между цивилизациями может происходить в форме "пересадки" (колонизации), "прививки" (ассимиляции) или "удобрения" (влияния на зарождающийся тип).
		\item О. Шпенглер ("Закат Европы"): Представил концепцию "морфологии истории". Он рассматривал мировые культуры как уникальные "организмы", обладающие собственной "душой". Каждая культура проходит предопределенный жизненный цикл: рождение (как пробуждение великой души), рост и созревание (в виде искусства, философии, государства) и неизбежную смерть, которая представляет собой переход в стадию "цивилизации" --- окостеневшей, бездушной, рациональной и технологичной формы, лишенной творческого потенциала.
		\item А. Тойнби ("Постижение истории"): Разработал теорию локальных цивилизаций. Движущей силой их развития является механизм \textbf{"Вызова-и-Ответа"}. Природные или социальные трудности (Вызов) ставят перед обществом проблему, и его дальнейшая судьба зависит от способности "творческого меньшинства" найти адекватный Ответ. Успешные ответы ведут к росту цивилизации. Если же общество не может дать ответ, оно надламывается и в конечном счете погибает.
		\item Э. Кассирер ("Философия символических форм"): Определил человека не как "animal rationale" (животное разумное), а как \textbf{"animal symbolicum"} (животное, создающее символы). Культура --- это совокупность символических систем (язык, миф, религия, искусство, наука), через которые человек организует, понимает и конструирует свой мир. Человек не может воспринимать реальность напрямую, а только через посредство этих символических форм.
	\end{itemize}
	
	\section{Исторический прогресс как проблема философии истории~\checkmark}
	\subsection{Концепции исторического развития}
	\begin{itemize}
		\item Линейная концепция прогресса (Просвещение, Гегель, Маркс): История рассматривается как поступательное, однонаправленное движение от низших форм к высшим, к единой для всего человечества цели.
		\begin{itemize}
			\item Просвещение (А. Тюрго, Ж.А. Кондорсе, И. Кант): Критерием прогресса является развитие разума, науки и просвещения, что должно привести к установлению справедливого общественного порядка и "вечному миру". Кант рассматривал историю как процесс постепенного осуществления правового устройства общества.
			\item Г.В.Ф. Гегель: История --- это "прогресс в сознании свободы". Мировой дух через деятельность народов и великих личностей познает самого себя, а конечной целью является реализация абсолютной свободы в разумном государстве.
			\item К. Маркс: Прогресс определяется развитием производительных сил, которое ведет к смене общественно-экономических формаций (первобытная, рабовладельческая, феодальная, капиталистическая). Конечная цель --- построение бесклассового коммунистического общества.
		\end{itemize}
		\item Циклическая концепция (Вико, Данилевский, Шпенглер, Тойнби): История --- это не единый процесс, а круговорот локальных, относительно замкнутых цивилизаций или культур. Каждая из них проходит схожие стадии зарождения, расцвета, упадка и гибели, после чего на историческую арену выходят новые народы.
		\begin{itemize}
			\item Джамбаттиста Вико: Считал, что каждая нация проходит три эпохи: век богов (теократия), век героев (аристократия) и век людей (демократия), после чего наступает упадок и возврат к варварству, дающий начало новому циклу.
		\end{itemize}
		\item Критика идеи прогресса (К. Поппер, постмодернизм): Отрицание существования универсальных законов и единой цели истории.
		\begin{itemize}
			\item Карл Поппер ("Нищета историцизма"): Критиковал "историцизм" (убежденность в существовании законов истории) как лженаучную и политически опасную доктрину, ведущую к тоталитаризму. Считал, что история не имеет предопределенного смысла, мы можем лишь придать ей цели (например, борьба за свободу и справедливость).
			\item Постмодернизм (Ж.-Ф. Лиотар): Идея прогресса --- это один из "великих метарассказов" (или "метанарративов") эпохи Модерна, наряду с идеями освобождения человечества или накопления знания. В эпоху Постмодерна эти большие повествования утратили свою легитимность и убедительность. История представляется как фрагментарный, нелинейный, хаотичный процесс без универсального смысла и вектора.
		\end{itemize}
	\end{itemize}
	\subsection{Критерии прогресса}
	В зависимости от философской позиции, в качестве критериев прогресса выдвигались:
	\begin{itemize}
		\item Развитие разума и науки: Рост научных знаний и их технологическое применение (концепция Просвещения).
		\item Нравственное совершенствование: Гуманизация общества, рост морального сознания (И. Кант, утопические социалисты).
		\item Технологическое развитие: Уровень производительных сил и степень господства человека над природой (марксизм).
		\item Рост свободы: Осознание и реализация свободы как главная цель истории (Гегель).
		\item Степень приближения к правовому устройству и "вечному миру" (И. Кант).
	\end{itemize}
	
	\section{Виртуалистика как направление современной философии~\checkmark}
	\subsection{Основные понятия}
	\begin{itemize}
		\item \textit{Виртуальная реальность}, \textit{киберпространство}: Искусственно созданная с помощью технических средств среда, передаваемая человеку через его ощущения и позволяющая осуществлять с ней взаимодействие.
		\item \textit{Симулякр} (Жан Бодрийяр): Ключевое понятие для описания современной культуры постмодерна. Это "копия без оригинала", знак, который не отсылает к какой-либо реальности, а лишь маскирует ее отсутствие. Бодрийяр выделял три порядка симулякров:
		\begin{enumerate}
			\item Копия/подделка (эпоха Ренессанса): Знак является отражением глубинной реальности (например, портрет).
			\item Серийное производство (промышленная революция): Знак маскирует и извращает реальность, копии становятся важнее оригинала.
			\item Симуляция: Знак маскирует отсутствие реальности, он предшествует "реальному" и порождает его.
		\end{enumerate}
		\item \textit{Гиперреальность} (Жан Бодрийяр): Состояние, в котором реальность вытесняется симулякрами, и становится невозможно отличить реальное от его симуляции. Медиа, реклама, информационные потоки создают мир, который кажется более реальным, чем сама реальность.
	\end{itemize}
	
	\subsection{Проблемное поле}
	\begin{itemize}
		\item Онтологический статус виртуальной реальности: Является ли виртуальность особой формой бытия, обладающей собственными законами, или лишь иллюзией, симуляцией "настоящей" реальности? Философы спорят о том, конструируется ли в сети новая реальность или происходит лишь имитация уже существующей.
		\item Проблема идентичности в сети: Киберпространство предоставляет беспрецедентные возможности для конструирования и "игры" с собственной идентичностью.
		\begin{itemize}
			\item Шерри Тёркл ("Жизнь на экране: идентичность в эпоху Интернета"): Рассматривает интернет как лабораторию для исследования "Я". Пользователи могут создавать множество аватаров, экспериментировать с гендером, возрастом и социальной ролью. Это приводит к "размыванию" и фрагментации личности, к представлению о "Я" не как о едином центре, а как о множестве гибких ролей.
			\item Возникают проблемы аутентичности, анонимности и ответственности за действия своих виртуальных "двойников".
		\end{itemize}
		\item Влияние на общество: Трансформация социальных связей, политики, экономики и культуры под воздействием цифровых технологий. Возникают новые формы сообществ (социальные сети, форумы), меняются способы коммуникации, политической мобилизации и ведения бизнеса.
	\end{itemize}
	
	\section{Синергетическая парадигма в современной философии~\checkmark}
	\subsection{Основные понятия и принципы}
	Синергетика (от греч. synergeia --- совместное действие) --- междисциплинарное направление, изучающее общие закономерности процессов самоорганизации в сложных неравновесных системах любой природы (физических, химических, биологических, социальных). Основоположники: Герман Хакен (ввел сам термин) и Илья Пригожин.
	\begin{itemize}
		\item \textit{Самоорганизация}: Процесс спонтанного возникновения упорядоченных структур в сложных системах без внешнего организующего воздействия.
		\item \textit{Нелинейность}: Свойство систем, при котором результат воздействия не пропорционален самому воздействию (малые причины могут вызывать огромные следствия и наоборот).
		\item \textit{Диссипативные структуры} (И. Пригожин): Открытые системы, далекие от термодинамического равновесия, которые обмениваются веществом и энергией с окружающей средой. Именно в таких системах возможна самоорганизация. Их существование поддерживается за счет постоянного рассеивания (диссипации) энергии.
		\item \textit{Бифуркация}: "Точка ветвления", критическое состояние системы, в котором она теряет устойчивость и возникает несколько альтернативных путей дальнейшего развития. Выбор одного из путей носит принципиально случайный, вероятностный характер и может зависеть от малых флуктуаций (случайных отклонений).
		\item \textit{Аттрактор}: Относительно устойчивое состояние (или структура), к которому стремится система в процессе своей эволюции. Аттрактор представляет собой "цель" движения системы.
	\end{itemize}
	
	\subsection{Применение в философии}
	Синергетический подход бросает вызов классическому лапласовскому детерминизму и линейному взгляду на развитие. Он применяется для анализа сложных самоорганизующихся систем, таких как общество, культура, история и познание.
	\begin{itemize}
		\item Философия истории: История может рассматриваться как нелинейный процесс, проходящий через точки бифуркации (революции, кризисы), где случайные события могут кардинально изменить траекторию развития общества.
		\item Эпистемология: Процесс научного познания и творческого мышления можно смоделировать как самоорганизацию, где рождение новой идеи (инсайт) является результатом перехода системы "знание" через точку неустойчивости.
		\item Социальная философия: Общество --- это сложная нелинейная система, эволюцию которой невозможно предсказать однозначно. Синергетика подчеркивает конструктивную роль хаоса и случайности в социальном развитии.
	\end{itemize}
	
	\section{Философия образования: основные проблемы~\checkmark}
	\subsection{Цели и ценности образования}
	\begin{itemize}
		\item Трансляция знаний и культуры: Передача накопленного человечеством социального и культурного опыта, знаний, норм и ценностей от поколения к поколению. Это обеспечивает преемственность и стабильность общества.
		\item Развитие личности: Формирование автономной личности, способной к критическому мышлению, творчеству, самоопределению и самореализации. Образование должно не просто "наполнять сосуд", но и "зажигать факел".
		\item Социализация и профессиональная подготовка: Подготовка индивида к жизни в обществе, усвоение им социальных норм и ролей, а также формирование компетенций, необходимых для успешной профессиональной деятельности.
	\end{itemize}
	
	\subsection{Основные модели образования}
	\begin{itemize}
		\item Традиционалистская (знаниевая, "банковская" по Фрейре): Основная цель --- передача ученику максимального объема готовых, систематизированных знаний. Учитель --- главный источник информации, ученик --- пассивный приемник. Акцент делается на запоминании и воспроизведении материала.
		\item Гуманистическая (личностно-ориентированная, педоцентристская): В центре образовательного процесса --- уникальная личность ученика, его потребности, интересы и способности. Цель --- не столько передача знаний, сколько развитие личности, ее творческого потенциала и стремления к самоактуализации.
		\begin{itemize}
			\item Джон Дьюи ("Демократия и образование"): Считал, что образование --- это не подготовка к жизни, а сама жизнь. Обучение должно строиться на основе личного опыта ребенка через решение практических проблем ("обучение через делание"). Школа --- это мини-общество, готовящее к участию в демократической жизни.
		\end{itemize}
		\item Компетентностная: Нацелена на формирование у учащихся набора ключевых компетенций --- то есть способности и готовности применять знания, умения и личностные качества для успешного решения практических и теоретических задач. Акцент смещается с "что я знаю" на "что я умею делать на основе того, что знаю".
		\item Критическая педагогика:
		\begin{itemize}
			\item Паулу Фрейре ("Педагогика угнетенных"): Критиковал "банковскую" модель образования, при которой учитель "вкладывает" знания в пассивных учеников. Он противопоставлял ей "проблемно-полагающее" образование, построенное на диалоге учителя и ученика. Цель такого образования --- развитие критического сознания ("conscientização"), которое позволяет людям осознать свое положение в обществе и стать субъектами его преобразования.
		\end{itemize}
	\end{itemize}
	
	
	\section{Основные проблемы биоэтики~\checkmark}
	Биоэтика --- область междисциплинарных исследований, направленная на осмысление, обсуждение и разрешение моральных проблем, порожденных новейшими достижениями в области медицины и биологических наук.
	
	\subsection{Проблема начала человеческой жизни}
	\begin{itemize}
		\item Моральный статус эмбриона: Центральный вопрос --- с какого момента человеческий эмбрион следует считать личностью, обладающей моральным статусом и правом на жизнь? Существуют разные позиции: от момента зачатия, с момента имплантации в стенку матки, с появлением нервной системы, с момента рождения и т.д.
		\item Этические проблемы абортов: Является ли аборт убийством или реализацией права женщины на распоряжение собственным телом? Дискуссия строится вокруг конфликта прав и интересов эмбриона/плода и беременной женщины.
		\item Вспомогательные репродуктивные технологии (ЭКО): Порождают проблемы "лишних" эмбрионов (их заморозка, уничтожение или использование в исследованиях), суррогатного материнства, донорства генетического материала.
	\end{itemize}
	
	\subsection{Проблема окончания жизни}
	\begin{itemize}
		\item Эвтаназия: Проблема права человека на добровольный уход из жизни в случае неизлечимого заболевания, сопровождающегося невыносимыми страданиями. Различают:
		\begin{itemize}
			\item Активная эвтаназия: Преднамеренное введение врачом препаратов, влекущих быструю и безболезненную смерть.
			\item Пассивная эвтаназия: Прекращение поддерживающей жизнь терапии по просьбе пациента.
		\end{itemize}
		Аргументы "за" апеллируют к автономии личности и праву на избавление от страданий. Аргументы "против" --- к святости человеческой жизни, клятве Гиппократа и опасности "скользкой дорожки" (злоупотреблений).
		\item Определение критериев смерти: Развитие реаниматологии привело к проблеме определения момента смерти. Констатация "смерти мозга" при работающем сердце и дыхании (поддерживаемых аппаратурой) ставит сложные этические и юридические вопросы, особенно в контексте трансплантологии.
	\end{itemize}
	
	\subsection{Генетическая инженерия и клонирование}
	\begin{itemize}
		\item Редактирование генома (напр., CRISPR-Cas9): Открывает перспективы лечения наследственных заболеваний, но вместе с тем ставит серьезные этические проблемы. Главные опасения связаны с редактированием генов в эмбрионах (в зародышевой линии), так как изменения будут передаваться по наследству. Это порождает риск создания "дизайнерских детей" с "улучшенными" характеристиками (интеллект, внешность), что может привести к новому виду евгеники и социальному расслоению.
		\item Клонирование человека: Обсуждается дилемма между потенциальной терапевтической пользой (выращивание органов для трансплантации) и этической недопустимостью репродуктивного клонирования. Основные аргументы против: нарушение уникальности человеческой личности, проблемы идентичности и психологического статуса клона, угроза инструментализации человека.
	\end{itemize}
	
	\section{Философское осмысление проблемы прав человека~\checkmark}
	Проблема прав человека — одна из центральных тем в современной политической, этической и правовой философии. Она включает в себя размышления о сущности человека, свободе, справедливости, достоинстве и законе. Философское осмысление прав человека требует обращения как к онтологическим основаниям человека как субъекта права, так и к социально-историческим формам борьбы за эти права. Это не просто нормативная проблема, а глубоко философская, так как связана с вопросами: что делает человека человеком?, на каком основании он обладает правами?, универсальны ли права человека?, можно ли их ограничивать и в каком объеме?
	
	\subsection{Античность и Средневековье: отсутствие категории <<права человека>>}
	В античной философии, несмотря на активное развитие этики и политики, понятие прав человека в современном смысле отсутствует. У Платона и Аристотеля основное внимание уделяется должному порядку, и права рассматриваются как статусные привилегии, зависящие от положения в полисе. Аристотель утверждает, что человек по природе <<политическое животное>>, и что его благо — это благо в составе целого. Индивид не рассматривается как носитель неотъемлемых прав: важна не личность, а функция в гармонии государства.
	
	В христианской философии Средневековья (Августин, Фома Аквинский) утверждается идея о божественном происхождении человеческого достоинства как творения по образу и подобию Бога. Однако акцент делается на обязанности, а не на правах. Права как юридически оформленные свободы — это продукт позднейшей эпохи. Тем не менее, здесь закладываются предпосылки для идеи универсального достоинства человека, независимо от его социального положения.
	
	\subsection{Новое время: зарождение философии прав человека}
	Философия Нового времени — эпоха, когда формируется современное понимание прав человека как естественных, неотъемлемых и универсальных.
	
	Томас Гоббс (1588–1679) в <<Левиафане>> утверждает, что в естественном состоянии человек имеет право на всё, включая жизнь других. Однако из страха перед насилием люди заключают договор и передают свои права суверену, чтобы обрести безопасность. Гоббс фактически признаёт право на жизнь как фундаментальное, но допускает его ограничение ради порядка. Таким образом, право здесь — результат воли и страха, а не достоинства.
	
	Джон Локк (1632–1704), напротив, формулирует классическую теорию естественных прав: каждый человек по природе обладает правом на жизнь, свободу и собственность, и целью государства является защита этих прав. Локк подчеркивает, что власть должна быть ограничена правом, а народ имеет право на восстание, если эти права нарушаются. Здесь права человека имеют морально-природное основание, независимое от государства.
	
	Жан-Жак Руссо в <<Общественном договоре>> утверждает, что человек рождается свободным, но в обществе оказывается в цепях. Его концепция <<общей воли>> стремится объединить свободу и закон: человек подчиняется закону, который он сам себе дал. Права человека, по Руссо, — это не просто свободы <<от>>, но и свободы <<для>> участия в общем волеобразовании. Он положил начало республиканской традиции прав, где важно не только ограничение власти, но и участие граждан в управлении.
	
	\subsection{Эпоха Просвещения и декларации прав}
	Эпоха Просвещения (XVIII век) — время теоретического и практического утверждения прав человека. Здесь права провозглашаются универсальными, основанными на разумах и достоинстве человека как морального существа.
	
	Иммануил Кант утверждает, что человек как разумное и автономное существо должен всегда рассматриваться как цель, а не как средство (категорический императив). Это означает, что права человека не зависят от контекста, культуры или пользы — они априорны, потому что основаны на достоинстве как трансцендентальной характеристике личности. Кант также вводит различие между моральными и юридическими правами, связывая их с понятием свободы в правовом государстве (Rechtsstaat).
	
	Практическим выражением этих идей становятся Декларация независимости США (1776) и Декларация прав человека и гражданина во Франции (1789). Обе утверждают, что права — это естественные, неотчуждаемые и принадлежащие каждому человеку с рождения свободы.
	
	\subsection{Критика и развитие идеи прав человека в XIX–XX веках}
	Карл Маркс критикует буржуазное понимание прав человека, считая их выражением эгоистических интересов частного собственника. В своей работе <<К еврейскому вопросу>> он противопоставляет права человека как изолированного индивида правам гражданина как члена политического сообщества. По Марксу, истинное освобождение возможно только при устранении частной собственности как источника отчуждения и социального неравенства.
	
	Фридрих Ницше выступает с радикальной критикой идеи равенства, считая её выражением <<рабской морали>> и <<слабости>>. В его философии ценность принадлежит не всем, а только сильному, творческому индивиду. Такая позиция обнажает антинатуралистические и иерархические основания критики прав человека.
	
	В XX веке философия прав человека получает развитие в неокантианстве, экзистенциализме, феноменологии и аналитической философии.
	
	Карл Ясперс и Габриэль Марсель утверждают, что права человека не могут быть поняты вне экзистенциального опыта свободы, ответственности и страдания.
	
	Ханна Арендт в работе <<Истоки тоталитаризма>> подчеркивает, что права человека становятся реальными только в политическом пространстве — там, где человек признан как гражданин. Она показывает, что беженцы, лишённые гражданства, теряют не только защиту, но и саму видимость принадлежности к человечеству. Следовательно, право иметь права — это фундаментальное условие прав человека.
	
	Юрген Хабермас предлагает концепцию прав как условий коммуникации. В рамках его дискурсивной этики права человека — это результат консенсуса в открытом и рациональном диалоге. Он подчеркивает, что права не только защищают индивида, но и формируют условия участия в общественном диалоге.
	
	\subsection{Современные философские споры о правах человека}
	Современная философия ведёт дебаты о универсальности, культурной обусловленности и иерархии прав.
	
	Культурный релятивизм (К.А. Аппиа, Дж. Строус) утверждает, что права человека не могут быть универсальными, так как разные культуры по-разному понимают свободу, достоинство, семью, личность.
	
	Универсализм (Марта Нуссбаум, А. Сен) утверждает, что, несмотря на культурные различия, существуют базовые условия человеческого достоинства (право на жизнь, телесную неприкосновенность, образование, участие в политике), которые должны быть защищены повсеместно.
	
	Проблема коллективных прав: современная философия ставит вопрос, возможно ли признание прав народов, меньшинств, культурных групп наряду с правами индивида, и не вступает ли это в противоречие с индивидуализмом либеральной традиции.
	
	\section{Философские аспекты проблемы искусственного интеллекта}
	
	\section{Философия экологии~\checkmark}
	\subsection{Основные направления}
	\begin{itemize}
		\item Антропоцентризм: Подход, согласно которому только человек обладает высшей внутренней ценностью, а природа рассматривается как ресурс для удовлетворения его потребностей, как объект эксплуатации и преобразования. Это мировоззрение, господствующее со времен промышленной революции.
		\item Биоцентризм (экоцентризм): Подход, утверждающий, что природа и все живые существа имеют собственную, внутреннюю ценность независимо от их пользы для человека. Человек рассматривается не как хозяин природы, а как один из членов биосферного сообщества.
		\begin{itemize}
			\item Глубинная экология (Арне Нэсс): Радикальная форма экоцентризма. Нэсс противопоставляет "поверхностную" экологию (борьба с загрязнением и истощением ресурсов в интересах человека) и "глубинную" , которая требует фундаментального пересмотра мировоззрения, ценностей и образа жизни современного индустриального общества.
		\end{itemize}
	\end{itemize}
	
	\subsection{Ключевые концепции}
	\begin{itemize}
		\item Устойчивое развитие: Модель развития, при которой удовлетворение потребностей нынешних поколений осуществляется без ущерба для возможностей будущих поколений удовлетворять свои собственные потребности. Эта концепция пытается найти баланс между экономическим ростом, социальной справедливостью и сохранением окружающей среды.
		\item Коэволюция человека и биосферы: Идея о необходимости гармоничного совместного развития общества и природы.
		\begin{itemize}
			\item Концепция ноосферы В. И. Вернадского: Ноосфера ("сфера разума") --- это новое, высшее состояние биосферы, преобразованной научной мыслью и трудом человечества. Вернадский считал, что человечество, став мощной геологической силой, должно взять на себя разумную ответственность за дальнейшую эволюцию биосферы, обеспечивая ее гармоничное развитие.
		\end{itemize}
	\end{itemize}
	
	\section{Историческая эволюция университета: университеты 1-го, 2-го и 3-го поколений~\checkmark}
	\subsection{Университет 1.0 (Средневековый, до-классический)}
	\begin{itemize}
		\item Функция: Хранение и трансляция знаний, в первую очередь теологических и канонических. Университет выступал как корпорация магистров и студентов, хранительница установленной истины.
		\item Образовательная модель: Схоластическая. Основана на изучении, комментировании и толковании авторитетных текстов (Священное Писание, труды Аристотеля, отцов Церкви). Главные методы --- лекция и диспут.
	\end{itemize}
	
	\subsection{Университет 2.0 (Классический / Гумбольдтовский)}
	\begin{itemize}
		\item Функция: Исследование и производство нового знания. Эта модель, предложенная Вильгельмом фон Гумбольдтом для Берлинского университета (1810), стала образцом для многих университетов мира.
		\item Ключевые принципы:
		\begin{itemize}
			\item Единство обучения и исследования: Студенты с самого начала вовлекаются в исследовательский процесс под руководством профессоров. Обучение --- это совместный поиск истины, а не просто передача готовых знаний.
			\item Академическая свобода: Свобода преподавания (профессор сам определяет содержание курса) и свобода учения (студент сам выбирает курсы и траекторию обучения).
			\item Университет как уединенное место: Идея "чистой" науки, свободной от прямого утилитарного давления со стороны государства или экономики.
		\end{itemize}
	\end{itemize}
	
	\subsection{Университет 3.0 (Предпринимательский, "вторая академическая революция")}
	\begin{itemize}
		\item Функция: Наряду с образованием и исследованиями, появляется \textbf{"третья миссия"} --- коммерциализация знаний, инновации, трансфер технологий и прямое участие в экономическом развитии региона и страны.
		\item Характеристики: Университет становится активным экономическим агентом. Он создает технопарки, бизнес-инкубаторы, патентует изобретения, выполняет заказы промышленных компаний. Акцент делается на прикладных исследованиях и подготовке специалистов, востребованных на рынке труда.
	\end{itemize}
	
	\subsection{Университет 4.0 (Университет в эпоху Четвертой промышленной революции)}
	\begin{itemize}
		\item Функция: Это модель университета, адаптированная к условиям цифровой экономики и Индустрии 4.0. Ее цель --- подготовка кадров с цифровыми компетенциями и создание инновационной среды, основанной на киберфизических системах.
		\item Характеристики:
		\begin{itemize}
			\item Цифровизация: Использование технологий искусственного интеллекта, больших данных, виртуальной и дополненной реальности для создания персонализированной и интерактивной образовательной среды.
			\item Проектное и сетевое обучение: Фокус на междисциплинарных проектах, часто выполняемых в сотрудничестве с другими университетами и реальными компаниями.
			\item Развитие талантов: Университет не просто передает знания, а становится платформой для развития талантов, креативности и предпринимательских навыков на протяжении всей жизни (lifelong learning).
		\end{itemize}
	\end{itemize}
	
	\section{Технологические уклады в истории общества~\checkmark}
	Технологический уклад — это исторически сложившаяся совокупность базовых технологий, форм организации производства, типов трудовых отношений и соответствующих им социальных структур. Каждому укладу соответствует определённый уровень развития производительных сил и специфическая социальная конфигурация. С философской точки зрения, смена технологических укладов — это не просто череда материальных изменений, но глубинный процесс трансформации отношения человека к природе, труду, самому себе и обществу.
	
	Понятие технологического уклада широко используется в современной философии техники, социальной философии и экономической теории. Особый вклад в его разработку внесли представители институциональной экономики (Н.Д. Кондратьев, К. Перес) и философии истории. Однако философское осмысление технологических укладов начинается значительно раньше — уже в трудах Фрэнсиса Бэкона, Гегеля, Маркса и других мыслителей, осмыслявших роль техники в преобразовании мира и общества.
	
	\subsection{Первый технологический уклад}
	связан с аграрным производством, основанным на ручном труде и примитивных орудиях. Это доиндустриальная, аграрная цивилизация. С философской точки зрения, человек здесь ещё в тесной зависимости от природы. Античная философия, например у Аристотеля, воспринимает труд как низшую деятельность, подчинённую цели, а созерцание — как высшую форму существования. Рабочий труд считался уделом рабов, а техника — лишь продолжением природы.
	
	\subsection{Второй технологический уклад}
	формируется в эпоху мануфактуры, начиная с позднего Средневековья и раннего Нового времени. Возникает механическая производственная система, базирующаяся на простом разделении труда. Здесь важен вклад философии Нового времени, в первую очередь Фрэнсиса Бэкона, который осмысляет науку и технику как средства господства над природой. Его девиз — <<знание — сила>> — символизирует поворот к активному преобразованию мира. Томас Гоббс продолжает эту линию, представляя общество как искусственно сконструированный механизм.
	
	\subsection{Третий технологический уклад}
	связанный с первой промышленной революцией (XVIII–XIX века), базируется на паровой энергии, угле, механизации и фабричном производстве. В философии этот период связан с зарождением социалистической мысли и критики капитализма. Карл Маркс и Фридрих Энгельс выдвигают идею, что техника и производственные отношения формируют основу общества (базис), а надстройка (идеология, культура) лишь выражает его интересы. Маркс подчёркивает отчуждающий характер труда в условиях индустриального производства и рассматривает технологический прогресс как противоречивый: он даёт возможность освобождения, но в рамках капитализма становится инструментом эксплуатации.
	
	\subsection{Четвёртый технологический уклад}
	соответствующий второй промышленной революции, характеризуется внедрением электричества, конвейерного производства и массового потребления. Здесь философия сталкивается с проблемой рационализации и бюрократизации общества. Макс Вебер вводит понятие <<железной клетки рациональности>>, подчёркивая, что технический прогресс ведёт к обезличенности, управляемости, стандартизации. Человек всё больше воспринимается как элемент системы, как функция.
	
	\subsection{Пятый технологический уклад}
	возникает в XX веке на основе электроники, телекоммуникаций, информационных технологий и автоматизации. На этом этапе философия техники получает своё институциональное оформление как отдельная дисциплина. Марин Хайдеггер в работе <<Вопрос о технике>> указывает на опасность технизации мышления: техника больше не является нейтральным инструментом, она определяет саму структуру восприятия и мышления. Франкфуртская школа (Адорно, Хоркхаймер) развивает эту критику, указывая на то, что технологическая рациональность подавляет человеческую свободу и творческое начало.
	
	\subsection{Шестой технологический уклад}
	который развивается сегодня, основан на цифровых технологиях, биоинженерии, искусственном интеллекте и нейросетевых платформах. Философы поднимают вопрос о границах человеческого. Трансгуманизм как философское направление (Ник Бостром, Рэй Курцвейл) утверждает, что человек больше не есть нечто завершённое, а представляет собой платформу для постоянной модернизации. В противовес этому выступают сторонники гуманистической традиции (например, Юваль Харари и Ханс Йонас), утверждая, что без этических ограничений технологическое развитие может подорвать саму идею человека как морального субъекта.
	
	\section{Промышленные революции в истории общества~\checkmark}
	\par Промышленные революции представляют собой качественные скачки в развитии производительных сил общества, сопровождающиеся глубокими социальными, экономическими, культурными и философскими изменениями. С философской точки зрения, эти процессы поднимают ряд фундаментальных вопросов: о сущности техники и технологии, об отношении человека к труду, о трансформации общественного устройства, о свободе и отчуждении в условиях индустриального и постиндустриального общества. Анализ промышленных революций требует обращения к идеям как социально-философского, так и онтологического и этического характера.
	\par Современная историко-философская наука выделяет четыре основных этапа промышленных революций.
	\subsection{Первая промышленная революция}
	\par Началась в конце XVIII века и была связана прежде всего с изобретением паровой машины, механизацией текстильной промышленности и широким использованием угля в качестве источника энергии. С философской точки зрения, эта эпоха опиралась на идеи Просвещения — философского направления, утверждавшего приоритет разума, науки и прогресса. Философы такие как Иммануил Кант, Жан-Жак Руссо и Вольтер подчеркивали освобождающую силу знания и видели в науке путь к построению более справедливого общества. Однако уже в этот период возникали первые критические интерпретации происходящих изменений. Так, утопические социалисты, среди которых Шарль Фурье и Анри де Сен-Симон, указывали на рост социального неравенства, вызванного концентрацией капитала и отчуждением труда, и предлагали альтернативные модели справедливого общественного устройства.
	\subsection{Вторая промышленная революция}
	\par пришедшаяся на вторую половину XIX и начало XX века, была связана с внедрением электричества, химической промышленности, массового производства и новых форм связи. В это время происходит важнейший философский поворот, связанный с критикой капиталистического способа производства. Карл Маркс, анализируя капитализм, показывает, что промышленное развитие не освобождает человека, а, напротив, усиливает отчуждение, делая его <<приложением к машине>>. Маркс формулирует идею научного социализма, полагая, что только радикальное преобразование общественных отношений может устранить фундаментальные противоречия капиталистической системы. Одновременно с этим философ Фридрих Ницше выдвигает критику рационализма и индустриального общества, подчеркивая его обезличенность и утрату творческого начала. Его концепция <<воли к власти>> становится реакцией на обездвиженность, подчинённость человека технической рациональности и массам.
	
	\subsection{Третья промышленная революция}
	начавшаяся в середине XX века, получила название научно-технической. Она характеризуется автоматизацией производства, широким применением электроники, развитием информационных технологий и атомной энергетики. На этом этапе философская мысль сталкивается с новыми вызовами. Мартин Хайдеггер в работе <<Вопрос о технике>> подчеркивает, что техника — это не просто совокупность орудий, но особый способ отношения к бытию, в котором природа и человек воспринимаются как ресурс. Он вводит понятие <<поставляющего мышления>>, в рамках которого утрачивается подлинное отношение к сущему. Представители Франкфуртской школы, такие как Макс Хоркхаймер, Теодор Адорно и Герберт Маркузе, акцентируют внимание на том, что технический прогресс не ведет к освобождению, а становится инструментом усиления контроля и манипуляции, формируя <<одномерного человека>> — лишенного способности к критическому мышлению.
	
	\subsection{Четвёртая промышленная революция}
	происходящая в настоящее время, связана с цифровыми технологиями, искусственным интеллектом, биоинженерией и роботизацией. Эти процессы ставят перед философией новые и радикальные вопросы: возможно ли сохранение человеческой идентичности в условиях алгоритмического управления и постоянного мониторинга? Где проходят границы между естественным и искусственным, между человеком и машиной? Трансгуманизм — философское течение, сформировавшееся в ответ на эти вызовы, предлагает рассматривать человека как проект, открытый к модификации и совершенствованию. Представители этого направления, например Ник Бостром и Рэй Курцвейл, полагают, что развитие технологий способно преодолеть традиционные человеческие ограничения, включая старение и смерть. Однако критики трансгуманизма — среди них Юваль Ной Харари и Бруно Латур — предупреждают об угрозах утраты автономии, усиления социального неравенства и девальвации человеческого достоинства.
	
	\par Каждое новое технологическое преобразование общества вызывает не только социально-экономические, но и глубоко антропологические и онтологические изменения. Вопрос о том, кем является человек в условиях технического прогресса, остается ключевым. Разные философские школы по-разному отвечают на этот вызов: от просветительского оптимизма до критической теории и постгуманистических сценариев.
	
	Если первая промышленная революция сопровождалась верой в прогресс и рациональность, то к началу XXI века философия всё чаще обращается к вопросам ограничений рациональности, этики технологий, сохранения субъективности и достоинства человека. Технологический прогресс, как показывают философские дебаты, не является нейтральным: он всегда ценностно нагружен и требует философского осмысления.
	
	\section{Глобальные проблемы в начале XXI века. Философия о будущем человечества}
	
\end{document}