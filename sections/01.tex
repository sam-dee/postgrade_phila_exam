\section{Предмет и специфика философского знания. Место философии в системе культуры}
Термин <<философия>> в переводе с древнегреческого означает любовь к мудрости, а с древ­неиндийского (<<даршака>>) — видение истины. Слово «философ» впервые употребил греческий математик и мыс­литель Пифагор по отношению к людям, стре­мящимся к интеллектуальному знанию и правильному образу жизни. 
\par Истолкование и закрепление в европейской культуре термина «фи­лософия» связано с именем Платона. Возникновение философии в VI в. до н. э. означало постепенный переход людей от мифа к самостоятельному размышлению о мире, о челове­ческой судьбе, стремление найти истину, стремление к мудрости. Поэтому философия выступала как «наука о всеобщем, которая исследует сущее само по себе, первые начала и причины» (Аристотель). 
\par  Философия — форма духовной деятельности, направленная на по­становку, анализ и решение коренных мировоззренческих вопросов, связанных с выработкой целостного взгляда на мир и на место чело­века в нем. В настоящее время философия представляет собой и науку о все­общих законах развития природы, общества, мышления, познания и особую форму общественного сознания, теоретическую основу ми­ровоззрения, систему философских дисциплин, способствующих фор­мированию духовного мира человека. 
\par Предметом философии называется круг вопросов, которые изуча­ет философия. Общую структуру предмета философии, философско­го знания составляют четыре основных раздела: онтология -- учение о бытии; гносеология (эпистемология) -- учение о познании; человек; общество.
\par По сравнению с другими формами знания, философия обладает следующими специфичными характеристиками:
\begin{enumerate}
	\item Фундаментальность — философия занимается первоосновами бытия и знания.
	\item Критичность — она подвергает сомнению даже очевидные основания (например, Декарт: "Cogito, ergo sum").
	\item Всеобщность — её суждения претендуют на универсальность.
	\item Полисемичность — философские понятия многозначны и подвержены интерпретации.
\end{enumerate}
Философия занимает особое место в системе культур человека, выступая
\begin{enumerate}
	\item как источник норм и ценностей (этика, политическая философия);
	\item как методологическое основание для других наук (эпистемология, философия науки);
	\item как историко-критическая форма сознания, осмысляющая культуру в целом.
\end{enumerate}
Философия повлияла на формирование естественных и гуманитарных наук (например, из философии выделились физика, социология, психология), на право (естественно-правовая теория), политику (от Платона до Хабермаса), искусство (эстетика) и даже религию (через философскую теологию).
